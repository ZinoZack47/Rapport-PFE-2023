%--------------A VOTRE ATTENTION-------------%
% Les étudiants  qui disposent de plus de 3 chapitres dans leurs travaux peuvent en complèter
% Les Membres doivent figurer dans la dernière version finale du mémoire pour dépôt de mémoire

\newcommand{\lang}{1}

\ifnum\lang=1
    \documentclass{iid}
\else
    \documentclass{iidEn}
\fi

\usepackage{titletoc}
\setlength{\glsdescwidth}{0.65\textwidth}
% \usepackage{lscape}

\ifnum\lang=1
    \typeMemoire{Diplôme d’Ingénieur d'État }
\optionFormation{\textbf{Informatique et Ingénierie des Données}\\ \includegraphics[scale=0.08]{images/logoIID.png}}
\etudiant{Ziyad \textbf{RAKIB}}
\titreDuMemoire{Participation à la migration d'une arichitecture monolithique à une arichitecture microservices} 
\dateSoutenance{15/06/2023}
%\promo{2\up{ème}}
\anneeScolaire{\the\year}


%%maitre de mémoire
\encadrants{Abdelghani \textbf{Ghazdali}\\Nidal \textbf{Lamghari}}

%% Membres du Jury
\jurys{%
\begin{tabular}{lll}
	Nom et prénoms du président &  Entité & Président \\
	Nom et prénoms de l'examinateur & Entité & Examinateur \\
	Nom et prénoms du rapporteur &  Entité & Rapporteur \\
	Nom et prénoms du rapporteur &  Entité & Rapporteur \\
\end{tabular}	
}
\else
    \thesisType{State Engineer Diploma}
\formationOption{\textbf{Computer Science and Data Engineering}\\ \includegraphics[scale=0.08]{images/logoIID.png}}
\student{Ziyad \textbf{RAKIB}}
\thesisTitle{Participation in the Migration from a Monolithic Architecture to a Microservices Architecture} 
\dateOfDefense{15/06/2023}
%\promo{2\up{ème}}
\schoolYear{\the\year}


%%maitre de mémoire
\tutors{Abdelghani \textbf{Ghazdali} (University)\\Nidal \textbf{Lamghari} (University)\\ Bilal \textbf{Slayki} (Organization)}

%% Membres du Jury
\jurys{%
\begin{tabular}{lll}
	OURDOU Amal     					     &  Entity & Chairman \\
	OUMMI Hssaine						     &  Entity & Examiner
\end{tabular}	
}
\fi


\hypersetup{
 pdftitle={--},
 pdfauthor={--},
 pdfsubject={--},
 pdfkeywords={--} 
 }

\color{bookColor}

%importation du glossaire
\loadglsentries{glossaire_reduit}

\begin{document}




\pageDeGarde%\pageTitre

\pagecolor{white}

%% page vide
\thispagestyle{empty}\ \clearpage
% ehrig2006graph
\newpage
% sommaire
\pagenumbering{roman}

\setcounter{tocdepth}{0}
\startlist{toc}
\printlist{toc}{}{\chapter*{Sommaire}}
\setcounter{tocdepth}{5}

\ifnum\lang=1
    %% rdedicaces
    \dedicace

\begin{fquote}
\begin{center}
\large{

\uppercase{à} ma mère, qui m'a comblée de son soutien et dévouée moi avec un amour inconditionnel. Tu es pour moi un exemple de courage et sacrifice continuel. Que cet humble travail porte témoigner de mon affection, de mon attachement éternel et qu'il appelle sur moi ta bénédiction continuelle.,\\[12pt]
\uppercase{à} mon père, aucune dédicace ne peut exprimer l'amour, l'estime, dévouement et le respect que j'ai toujours eu pour vous. Rien dans le monde vaut les efforts faits jour et nuit pour mon éducation et mon bien-être. Ce travail est le fruit de vos sacrifices que vous avez fait pour mon éducation et ma formation,\\[12pt]
\uppercase{à} mes chers frères, merci pour votre amour, soutien, et encouragements,\\[12pt]
\uppercase{à} tous mes chers amis, pour le soutien que vous m'avez apporté, je dis Merci encore une fois à tous ceux qui me sont chers, à vous tous\\[12pt]
Merci.
}
\end{center}
\bigskip
\medskip
\end{fquote}

\begin{adjustwidth}{2cm}{1cm}
\hspace*{\fill} \textbf{\textit{\large{- Ziyad}}}
\end{adjustwidth}

\clearpage

    \newpage 

    %% remerciements
    \remerciements

Tout d’abord, je remercie le grand Dieu puissant de nous donné la puissance pour continuer et dépasser toutes les difficultés.

\medskip

J’adresse mes remerciements à, \textbf{\LR{\large{Monsieur Abdelghani Ghazdali}}}, chef de filière Informatique et Ingénierie des Données qui a su assurer le bon déroulement des séances d’encadrement du début à la fin.

\medskip

Je tiens également à adresser mes sincères remerciements à \textbf{\LR{\large{Professeur Nidal Lamghari}}}, mon encadrante interne, pour tous les conseils qu’elle m’a prodigués ainsi que ses encouragements.

\medskip

Au terme de ce travail, je tiens également à exprimer mes sincères remerciements et ma gratitude envers tous ceux qui, par leur enseignement, par leur soutien et leurs conseils, ont contribué au déroulement de ce projet.

\medskip

J’adresse mes remerciements à \textbf{\LR{\large{Monsieur Reda EL Mejad}}}, PDG et co-fondateur de Izicap et lauréat de l’ENSA, pour l’opportunité de passer mon stage au sein de cet entreprise.

\medskip

Je tiens à remercier mon encadrant à Izicap, \textbf{\LR{\large{Monsieur Bilal SLAYKI}}} qui m’a supervisée avec patience et n’a épargné aucun effort pour mettre à ma disposition les explications nécessaires et les directives précieuses. Son assistance et ses conseils permanents m’ont été un apport remarquable. Je veux remercier aussi Nasr, Youness et Anas; l’équipe avec laquelle j’ai travaillé et qui ont généreusement contribué à ce travail.

\medskip

Je remercie également l'agente RH \textbf{\LR{\large{Madame Sanaa ARROUCHE}}} pour ses bons conseils car elle est toujours là pour accéder à nos demandes et répondre au mieux à nos questions.

\medskip

Un remerciement particulier aux membres du jurys qui m’ont honoré en acceptant de juger ce travail et de me faire profiter de leurs remarques et conseils.

\medskip

Finalement, mes remerciements s’adressent aussi à l’ensemble du corps professoral et administratif de l’ENSA Khouribga pour l’effort qu’ils fournissent afin de nous garantir une bonne formation et à l’équipe administrative et technique pour tous les services offerts.
    \newpage

    % Résume
    \resume
%\selectlanguage{french}

Le présent rapport de projet de fin d'études (PFE) met en évidence les différentes étapes de notre projet de transformation de l'architecture monolithique en Groovy vers une architecture basée sur des microservices en Java, tout en effectuant une transition de AngularJS vers React et en adoptant une approche de microfrontend. Nous avons entrepris cette démarche afin d'améliorer la flexibilité, la maintenabilité, les performances, la cohérence des données et la modularité de notre application.
\medskip

Initialement, notre code monolithique était développé en Groovy. La première étape de notre projet a consisté à analyser cette architecture monolithique et à identifier les composants qui pourraient être isolés en microservices indépendants. Parallèlement, nous avons évalué les avantages de migrer de AngularJS, un framework JavaScript obsolète, vers React, un framework plus moderne et performant.

\medskip
Nous avons procédé à la conception et à la mise en œuvre de ces microservices en utilisant SpringBoot, en veillant à découpler les fonctionnalités et à assurer une communication efficace entre les services. Dans le même temps, nous avons migré progressivement notre code AngularJS vers React, en réécrivant les fonctionnalités existantes avec les meilleures pratiques de développement React.
\medskip

Pour assurer la cohérence des données entre les microservices, nous avons mis en place Delta Lake, une technologie de gestion de données incrémentielle. Delta Lake a permis de garantir la fiabilité des données et de simplifier les opérations de lecture et d'écriture sécurisées entre les services.

\medskip
Les connecteurs en Java ont facilité la communication entre le frontend basé sur React et Delta Lake, assurant ainsi une connexion robuste et sécurisée. Les connecteurs ont également permis des opérations de lecture et d'écriture efficaces, en garantissant l'intégrité et la cohérence des données.

\medskip
 
En parallèle, nous avons également adopté une approche de microfrontend pour notre architecture frontend. Cela nous a permis de découpler les fonctionnalités du frontend en modules autonomes, offrant ainsi une plus grande flexibilité et la possibilité de les développer et de les déployer indépendamment.
\medskip

Enfin, pour faciliter le déploiement et la gestion de notre architecture basée sur des microservices, React et le microfrontend, nous avons adopté l'utilisation de Kubernetes et Docker. Ces outils ont permis l'orchestration et la mise à l'échelle efficaces des services, tout en simplifiant le déploiement dans différents environnements.

\noindent\rule[2pt]{\textwidth}{0.5pt}
{\textbf{Mots-clés:}}
Monolithique, Groovy, architecture microservices, AngularJS, React, Delta Lake, Microfrontend, Kubernetes, Docker.
\\
\noindent\rule[2pt]{\textwidth}{0.5pt}

\clearpage


    \newpage

    % Résume
    \resumeAn
%\begin{abstract}

This report provides an in-depth analysis of the complex project undertaken to scale the data source of Izicap. The project required the removal of MariaDB and implementation of Delta Lake, a high-performance data storage and processing solution. In addition, the existing monolithic architecture was divided into microservices utilizing Spring Boot as the backend with a suitable connector to Delta Lake, and ReactJS micro-frontends as the frontend.

\medskip

The project presented numerous challenges that required the application of advanced technologies and strategies. These included data migration, ensuring data consistency and accuracy, managing the complexities of distributed systems, and integrating various technologies and services. The report outlines the various solutions developed to address these challenges, including the use of advanced data processing algorithms, distributed computing architectures, and containerization.

\medskip

Despite the challenges encountered, the project remains a work in progress. The report provides insights into the ongoing efforts to improve the project's scalability, performance, and overall functionality.

\vspace{1cm}



\noindent\rule[2pt]{\textwidth}{0.5pt}

{\textbf{Keywords:}}
Project, Data Source, Delta Lake, Monolith, Microservice, Springboot, ReactJS, Micro-frontends, Back-End, Front-End, Data, Scalability, Performance.
\\
\noindent\rule[2pt]{\textwidth}{0.5pt}

    \newpage

    % Résume
    %\resumeAr
\setcode{utf8}

\chapter*{\hfill \RL{ملخص}}




\setstretch{1.3}
\begin{flushright}
\RL{
يقدم هذا التقرير تحليلاً متعمقًا للمشروع المعقد الذي تم إجراؤه لتوسيع نطاق مصدر بيانات الشركة. يتطلب المشروع إزالة \LR{MariaDB} وتنفيذ \LR{Delta Lake} ، وهو حل تخزين ومعالجة بيانات عالي الأداء. بالإضافة إلى ذلك ، تم تقسيم المونوليث الحالي إلى خدمات مصغرة باستخدام \LR{Spring Boot} كواجهة خلفية مع موصل مناسب لـ \LR{Delta Lake} ، وواجهة \LR{ReactJS} كواجهة أمامية.
قدم المشروع العديد من التحديات التي تطلبت تطبيق التقنيات والاستراتيجيات المتقدمة. وشمل ذلك ترحيل البيانات ، وضمان اتساق البيانات ودقتها ، وإدارة تعقيدات الأنظمة الموزعة ، ودمج التقنيات والخدمات المختلفة. يحدد التقرير الحلول المختلفة التي تم تطويرها لمواجهة هذه التحديات ، بما في ذلك استخدام خوارزميات معالجة البيانات المتقدمة ، وبنى الحوسبة الموزعة ، والحاويات.
على الرغم من التحديات التي واجهنها ، لا يزال المشروع قيد التنفيذ. يقدم التقرير رؤى حول الجهود الجارية لتحسين قابلية المشروع للتوسع والأداء والوظائف العامة
}
\end{flushright}


\vspace{1cm}


\noindent\rule[2pt]{\textwidth}{0.5pt}
\begin{flushright}
\RL{\textbf{
كلمات مفتاحية \LR{:} } المشروع، مصدر البيانات، \LR{Delta Lake}، المونوليث، الميكروسيرفس، \LR{Springboot}، \LR{ReactJS}، الميكرو-واجهات، الخلفية، الواجهة الأمامية، البيانات، القابلية للتوسع، الأداء. \LR{Delta Lake}، \LR{Springboot}، \LR{ReactJS}، الميكرو-واجهـات. 
}

\end{flushright}
\noindent\rule[2pt]{\textwidth}{0.5pt}

%\end{abstract}







    \newpage

    \selectlanguage{french}
\else
    %% rdedicaces
    \dedicace

\begin{fquote}
\begin{center}
\large{

\uppercase{To} my mother, who showered me with her support and devoted me with unconditional love. You are for me an example of courage and continuous sacrifice. May this humble work bear witness to my affection, my eternal attachment and may it call upon me your continual blessing,\\[12pt]
\uppercase{To} my father, no dedication can express the love, esteem, devotion, and respect I have always had for you. Nothing in the world is worth the efforts made day and night for my education and my well-being. This work is the fruit of your sacrifices that you made for my education and training,\\[12pt]
\uppercase{To} my dear brothers, thank you for your love, support, and encouragement,\\[12pt]
\uppercase{To} all my dear friends, for the support you have given me, I say\\[12pt]
Thank you.
}
\end{center}
\bigskip
\medskip
\end{fquote}

\begin{adjustwidth}{2cm}{1cm}
\hspace*{\fill} \textbf{\textit{\large{- Ziyad}}}
\end{adjustwidth}

\clearpage

    \newpage 

    %% remerciements
    \remerciements

First of all, I thank God Almighty for giving us the power to continue and overcome all the difficulties.
\medskip

I would like to thank \textbf{\LR{\large{Mr.Abdelghani Ghazdali}}}, head of the Computer Science and Data Engineering department who was able to ensure the smooth running of the coaching sessions from start to finish.

\medskip

I would also like to express my sincere thanks to \textbf{\LR{\large{Professor Nidal Lamghari}}}, my internal supervisor, for all the advice she gave me and her encouragement.

\medskip

During this work, I would also like to express my sincere thanks and gratitude to all those who, through their teaching, support and advice, have contributed to the development of this project.

\medskip

I would like to thank \textbf{\LR{\large{Mr.Reda EL Mejad}}}, CEO and co-founder of Izicap and ENSA laureate, for the opportunity to spend my internship within this company.

\medskip

I would like to thank my supervisor at Izicap, \textbf{\LR{\large{Mr.Bilal SLAYKI}}} who patiently supervised me and spared no effort to provide me with the necessary explanations and valuable guidelines. His assistance and prominent advice have been a remarkable contribution to my growth. I also want to thank Nasr, Youness and Anas; the team with whom I worked and who generously are contributing to this work.

\medskip

I would also like to thank the HR agent \textbf{\LR{\large{Madame Sanaa ARROUCHE}}} for her good advice as she is always there to respond to our requests and answer our questions as well as assisting us to the best of her ability.

\medskip

A special thank you to the members of the juries who honored me by accepting to judge this work and to give me the benefit of their comments and advice.

\medskip

Finally, my thanks also go to the entire faculty and administrative staff of ENSA Khouribga for the effort they provide to guarantee us good training and to the administrative and technical team for all the services offered.
    \newpage

    % Résume
    \resume
%\selectlanguage{french}

Le présent rapport de projet de fin d'études (PFE) met en évidence les différentes étapes de notre projet de transformation de l'architecture monolithique en Groovy vers une architecture basée sur des microservices en Java, tout en effectuant une transition de AngularJS vers React et en adoptant une approche de microfrontend. Nous avons entrepris cette démarche afin d'améliorer la flexibilité, la maintenabilité, les performances, la cohérence des données et la modularité de notre application.
\medskip

Initialement, notre code monolithique était développé en Groovy. La première étape de notre projet a consisté à analyser cette architecture monolithique et à identifier les composants qui pourraient être isolés en microservices indépendants. Parallèlement, nous avons évalué les avantages de migrer de AngularJS, un framework JavaScript obsolète, vers React, un framework plus moderne et performant.

\medskip
Nous avons procédé à la conception et à la mise en œuvre de ces microservices en utilisant SpringBoot, en veillant à découpler les fonctionnalités et à assurer une communication efficace entre les services. Dans le même temps, nous avons migré progressivement notre code AngularJS vers React, en réécrivant les fonctionnalités existantes avec les meilleures pratiques de développement React.
\medskip

Pour assurer la cohérence des données entre les microservices, nous avons mis en place Delta Lake, une technologie de gestion de données incrémentielle. Delta Lake a permis de garantir la fiabilité des données et de simplifier les opérations de lecture et d'écriture sécurisées entre les services.

\medskip
Les connecteurs en Java ont facilité la communication entre le frontend basé sur React et Delta Lake, assurant ainsi une connexion robuste et sécurisée. Les connecteurs ont également permis des opérations de lecture et d'écriture efficaces, en garantissant l'intégrité et la cohérence des données.

\medskip
 
En parallèle, nous avons également adopté une approche de microfrontend pour notre architecture frontend. Cela nous a permis de découpler les fonctionnalités du frontend en modules autonomes, offrant ainsi une plus grande flexibilité et la possibilité de les développer et de les déployer indépendamment.
\medskip

Enfin, pour faciliter le déploiement et la gestion de notre architecture basée sur des microservices, React et le microfrontend, nous avons adopté l'utilisation de Kubernetes et Docker. Ces outils ont permis l'orchestration et la mise à l'échelle efficaces des services, tout en simplifiant le déploiement dans différents environnements.

\noindent\rule[2pt]{\textwidth}{0.5pt}
{\textbf{Mots-clés:}}
Monolithique, Groovy, architecture microservices, AngularJS, React, Delta Lake, Microfrontend, Kubernetes, Docker.
\\
\noindent\rule[2pt]{\textwidth}{0.5pt}

\clearpage


    \newpage

    % Résume
    \resumeAn
%\begin{abstract}

This report provides an in-depth analysis of the complex project undertaken to scale the data source of Izicap. The project required the removal of MariaDB and implementation of Delta Lake, a high-performance data storage and processing solution. In addition, the existing monolithic architecture was divided into microservices utilizing Spring Boot as the backend with a suitable connector to Delta Lake, and ReactJS micro-frontends as the frontend.

\medskip

The project presented numerous challenges that required the application of advanced technologies and strategies. These included data migration, ensuring data consistency and accuracy, managing the complexities of distributed systems, and integrating various technologies and services. The report outlines the various solutions developed to address these challenges, including the use of advanced data processing algorithms, distributed computing architectures, and containerization.

\medskip

Despite the challenges encountered, the project remains a work in progress. The report provides insights into the ongoing efforts to improve the project's scalability, performance, and overall functionality.

\vspace{1cm}



\noindent\rule[2pt]{\textwidth}{0.5pt}

{\textbf{Keywords:}}
Project, Data Source, Delta Lake, Monolith, Microservice, Springboot, ReactJS, Micro-frontends, Back-End, Front-End, Data, Scalability, Performance.
\\
\noindent\rule[2pt]{\textwidth}{0.5pt}

    \newpage

    % Résume
    %\resumeAr
\setcode{utf8}

\chapter*{\hfill \RL{ملخص}}




\setstretch{1.3}
\begin{flushright}
\RL{
يقدم هذا التقرير تحليلاً متعمقًا للمشروع المعقد الذي تم إجراؤه لتوسيع نطاق مصدر بيانات الشركة. يتطلب المشروع إزالة \LR{MariaDB} وتنفيذ \LR{Delta Lake} ، وهو حل تخزين ومعالجة بيانات عالي الأداء. بالإضافة إلى ذلك ، تم تقسيم المونوليث الحالي إلى خدمات مصغرة باستخدام \LR{Spring Boot} كواجهة خلفية مع موصل مناسب لـ \LR{Delta Lake} ، وواجهة \LR{ReactJS} كواجهة أمامية.
قدم المشروع العديد من التحديات التي تطلبت تطبيق التقنيات والاستراتيجيات المتقدمة. وشمل ذلك ترحيل البيانات ، وضمان اتساق البيانات ودقتها ، وإدارة تعقيدات الأنظمة الموزعة ، ودمج التقنيات والخدمات المختلفة. يحدد التقرير الحلول المختلفة التي تم تطويرها لمواجهة هذه التحديات ، بما في ذلك استخدام خوارزميات معالجة البيانات المتقدمة ، وبنى الحوسبة الموزعة ، والحاويات.
على الرغم من التحديات التي واجهنها ، لا يزال المشروع قيد التنفيذ. يقدم التقرير رؤى حول الجهود الجارية لتحسين قابلية المشروع للتوسع والأداء والوظائف العامة
}
\end{flushright}


\vspace{1cm}


\noindent\rule[2pt]{\textwidth}{0.5pt}
\begin{flushright}
\RL{\textbf{
كلمات مفتاحية \LR{:} } المشروع، مصدر البيانات، \LR{Delta Lake}، المونوليث، الميكروسيرفس، \LR{Springboot}، \LR{ReactJS}، الميكرو-واجهات، الخلفية، الواجهة الأمامية، البيانات، القابلية للتوسع، الأداء. \LR{Delta Lake}، \LR{Springboot}، \LR{ReactJS}، الميكرو-واجهـات. 
}

\end{flushright}
\noindent\rule[2pt]{\textwidth}{0.5pt}

%\end{abstract}







    \newpage

    \selectlanguage{english}
\fi


\dominitoc% initializer les minitoc
\tableofcontents
\newpage

%liste des figures
\listoffigures 
\newpage

%liste des tableaux
\listoftables
\newpage

%%liste des algo
%\listofalgorithmes
%\newpage
%


%% Les sigles et acronymes
\begin{longtable}{rl}
\caption{List of Abbreviations}\label{tab:abbreviations} \\
API & Application programming interface \\
VSC & Visual Studio Code \\
DL & Delta Lake \\
REST & RepresEntational State Transfer \\
AWS & Amazon Web Services \\
GCP & Google Cloud Platform \\
AZR & Microsoft Azure \\
S3 & Simple Storage Service \\
IAM & Identity and Access Management \\
MinIO & Minimal Object Storage \\
SQL & Structured Query Language \\
DB & Database \\
ACID & Atomicity, Consistency, Isolation, Durability \\
OLTP & Online Transaction Processing \\
OLAP & Online Analytical Processing \\
ReactJS & React JavaScript \\
SPA & Single Page Application \\
JS & JavaScript \\
CSS & Cascading Style Sheets \\
HTML & Hypertext Markup Language \\
UI & User Interface \\
API & Application Programming Interface \\
Spark & Apache Spark \\
Hadoop & Apache Hadoop \\
HDFS & Hadoop Distributed File System \\
YARN & Yet Another Resource Negotiator \\
MapReduce & MapReduce Programming Model \\
Metadata & Data about Data \\
ETL & Extract, Transform, Load \\
ELT & Extract, Load, Transform \\
CRM & Customer Relationship Management \\
RDBMS & Relational Database Management System \\
SPI & Server Provider Interface \\
\end{longtable}

\newpage

%% Les sigles et acronymes
%\setglossarystyle{altlist}
%\printglossary[title=Liste des acronymes, toctitle=Liste des acronymes, type=\acronymtype]
%\newpage

% Le glossaire proprement dit
%\setglossarystyle{super}
%\printglossary[type=main]


\pagenumbering{arabic}
\setcounter{page}{1}
%%introduction
\introduction
Dans un monde numérique en constante évolution, les entreprises sont confrontées à des défis majeurs pour maintenir leurs applications à la pointe de la technologie et répondre aux attentes croissantes des utilisateurs. L'architecture monolithique présente des limitations en termes de flexibilité, de maintenabilité, de performances et de modularité, ce qui pousse de nombreuses organisations à adopter des architectures basées sur des microservices.

\medskip

Ce rapport met en évidence la démarche entreprise pour transformer une architecture monolithique en Groovy vers une architecture basée sur des microservices en utilisant Java, tout en migrant de AngularJS vers React et en adoptant une approche de microfrontend. L'objectif principal de cette transformation était d'améliorer la flexibilité, la maintenabilité, les performances, la cohérence des données et la modularité de l'application.

\medskip

La première étape a été d'analyser l'architecture monolithique existante et d'identifier les composants pouvant être isolés en microservices indépendants. En parallèle, une évaluation des avantages de migrer de AngularJS vers React a été réalisée.

\medskip

La conception et la mise en œuvre des microservices ont été réalisées en utilisant Spring-Boot, en veillant à découpler les fonctionnalités et à assurer une communication efficace entre les services. Le code AngularJS a été progressivement migré vers React en utilisant les meilleures pratiques de développement React.

\medskip

Pour garantir la cohérence des données entre les microservices, la technologie de gestion de données incrémentielle Delta Lake a été utilisée. Des connecteurs en Java ont facilité la communication entre le frontend basé sur React et Delta Lake, assurant une connexion robuste et sécurisée.

\medskip

En parallèle, une approche de microfrontend a été adoptée pour le frontend, permettant de découpler les fonctionnalités en modules autonomes.

\medskip

Ce rapport détaillera chaque étape de la transformation, mettant l'accent sur les décisions prises, les défis rencontrés et les résultats obtenus. Il soulignera les avantages de l'architecture basée sur des microservices, l'utilisation de React et du microfrontend, ainsi que l'utilisation de Kubernetes et Docker. Cette étude de cas offre une perspective précieuse sur la modernisation des applications et les meilleures pratiques de développement dans un environnement en constante évolution.
%\lhead[]{} \rhead[]{} \chead[]{}
\ifnum\lang=1
    \selectlanguage{french}
\else
    \selectlanguage{english}
\fi
\fancyhead[L]{\tiny \leftmark}
\fancyhead[R]{\scriptsize \rightmark}
\fancyfoot[C]{\thepage}


\ifnum\lang=1
    \chapter{Contexte général du projet}\label{chap:1}
    \minitoc
\addcontentsline{toc}{section}{Introduction}
\section{Introduction}
L'objectif de cette partie est de présenter le contexte général du projet, en commençant par l'organisme d'accueil, IZICAP, où je suis en train d'effectuer un stage de 5 mois. Nous décrirons leurs activités, leur stratégie, leurs valeurs, ainsi que l'organigramme et les profils impliqués dans mon projet. Ensuite, nous aborderons la demande de travaux qui exprime les besoins fonctionnels sur lesquels je me suis basé pour concevoir et réaliser la solution, ainsi que la méthodologie suivie pour atteindre les objectifs fixés.

\section{Présentation de l’organisme d’accueil}
\subsection{IZICAP}
IZICAP est une société unique et novatrice axée sur l'analyse de données, la fidélisation et le marketing numérique pour les petites et moyennes entreprises (PME). Leur solution conviviale et facile à installer repose sur le modèle SaaS, permettant d'exploiter les données transactionnelles des cartes de paiement des clients.

En collaborant avec les banques et les acquéreurs, IZICAP offre une proposition de valeur renforcée et mettons à leur disposition un outil puissant de marketing numérique destiné aux PME. Leur mission est d'accompagner les commerçants au quotidien pour accélérer le développement de leur entreprise, répondant ainsi à leurs besoins spécifiques.

L'histoire d'IZICAP remonte à 2013, lorsque elle a vu le jour dans un paysage des services financiers en constante évolution et de plus en plus complexe. Depuis lors, les modèles bancaires et de paiement traditionnels sont continuellement menacés par de nouvelles entreprises challengers plus innovantes. Une concurrence agressive signifie que la fidélité des commerçants ne peut plus être considérée comme acquise, et souvent les acquéreurs et les fournisseurs de services de paiement sont contraints de réduire leurs frais.

Après s'être solidement implantée en France grâce à des partenariats avec le Groupe BPCE et le Crédit Agricole, Izicap a établi un partenariat avec Nexi, le principal acquéreur et fintech en Italie, et a rejoint le programme StartPath de Mastercard dans le but d'étendre considérablement sa portée à l'échelle mondiale. Cette expansion stratégique renforce Leur position sur le marché et confirme Leur engagement à offrir leur services à un public plus large.

Avec Izicap, les PME ont accès à une solution innovante qui tire parti de l'analyse de données pour optimiser leur stratégie de fidélisation et de marketing. Grâce à leur expertise et à leur présence internationale croissante, IZICAP est déterminée à soutenir la croissance des entreprises et à les aider à prospérer dans un environnement commercial en constante évolution.

\cite{izicap}

\subsection{Partenariats stratégiques}
IZICAP a établi des partenariats stratégiques avec plusieurs acteurs clés de l'industrie financière. Parmi ses partenaires notables figurent :
\begin{itemize}
    \item \textbf{Groupe BPCE:} Izicap a collaboré avec le Groupe BPCE, l'un des principaux groupes bancaires en France, pour fournir sa solution de fidélisation et de marketing numérique aux commerçants.
    \item \textbf{Crédit Agricole:} Izicap a établi un partenariat avec Crédit Agricole, l'une des plus grandes banques françaises, afin d'offrir ses services aux commerçants de leur réseau.
    \item \textbf{Nexi:} Izicap a noué un partenariat avec Nexi, le principal acquéreur et fintech en Italie. Cette collaboration permet à Izicap d'étendre sa présence sur le marché italien et de bénéficier de l'expertise de Nexi dans le domaine des services de paiement.
    \item \textbf{Mastercard StartPath:} Izicap a rejoint le programme StartPath de Mastercard, qui soutient les start-ups et les entreprises innovantes dans le domaine des technologies financières. Cette collaboration offre à Izicap l'opportunité d'accroître sa visibilité et son expansion internationale.
    \item \textbf{BBVA:} Izicap a également établi un partenariat avec BBVA, l'une des principales banques en Espagne et en Amérique latine. Cette collaboration avec BBVA renforce la présence d'Izicap sur le marché espagnol et permet à la société d'offrir ses solutions de fidélisation et de marketing numérique aux commerçants affiliés à BBVA. Le partenariat avec BBVA témoigne de l'engagement d'Izicap à étendre sa portée et à travailler avec des acteurs majeurs de l'industrie financière pour soutenir les petites et moyennes entreprises dans leur croissance.
    \item \textbf{Ingenico:} Izicap a établi un partenariat avec Ingenico, l'un des principaux fournisseurs de solutions de paiement et de services marchands au niveau mondial. Grâce à ce partenariat, Izicap est en mesure d'intégrer sa solution de fidélisation et de marketing numérique aux terminaux de paiement d'Ingenico, offrant ainsi une solution complète aux commerçants utilisant les services d'Ingenico. Cette collaboration renforce la position d'Izicap en tant que fournisseur de solutions de fidélisation et de marketing innovantes, en s'appuyant sur l'expertise et la portée mondiale d'Ingenico dans le secteur des paiements.
\end{itemize}
\begin{figure}[H]
\centering
\includegraphics[width=\linewidth]{images/partenaires.png}
\caption{Partenariats stratégiques}\label{fig:partenaires}
\end{figure}    
\section{Organigramme}
\begin{figure}[H]
\centering
\includegraphics[width=0.75\linewidth]{images/organigram.png}
\caption{Organigramme de Izicap}\label{fig:organigmramme}
\end{figure}

Au sein de l'équipe de projet, j'ai eu l'opportunité de travailler de manière polyvalente dans différents domaines. J'ai contribué activement au développement du backend en participant à la conception et à l'implémentation des microservices. J'ai également collaboré avec l'équipe frontend en apportant mon expertise pour l'intégration des microfrontends et l'amélioration de l'expérience utilisateur. De plus, j'ai été impliqué dans la gestion des données en participant à la mise en place de Delta Lake et en contribuant à l'optimisation des flux de données. Cette expérience diversifiée m'a permis d'acquérir une vision globale du projet et de collaborer efficacement avec différentes équipes pour atteindre les objectifs fixés.

\section{Cadre du projet}

Le projet consiste à refonter l'application `Smart data' qui fait référence au processus d'extraction d'informations précieuses à partir de grandes quantités de données afin de prendre des décisions commerciales éclairées.

La smart data nous permet de:

\begin{enumerate}
    \item \textbf{Collecte des données}: Izicap aide les entreprises à collecter des données clients à partir de différents points de contact tels que les systèmes de point de vente, les programmes de fidélité, les interactions en ligne, etc. Assurez-vous d'intégrer leurs outils de collecte de données dans vos systèmes ou processus existants.
    \item \textbf{Analyse des données}: La solution de smart data d'Izicap vous permet d'analyser les données collectées pour découvrir des tendances, des modèles et des comportements clients. Utilisez leurs outils d'analyse et leurs algorithmes pour obtenir des informations exploitables à partir des données.
    \item \textbf{Segmentation des clients}: Segmentez votre base de clients en fonction de leurs préférences, de leur historique d'achat, de leurs caractéristiques démographiques ou d'autres critères pertinents. La solution de smart data d'Izicap peut vous aider à identifier différents segments de clients et à créer des stratégies marketing personnalisées pour chaque segment.
    \item \textbf{Marketing personnalisé}: Exploitez les informations tirées de la solution de smart data d'Izicap pour créer des campagnes marketing ciblées. Envoyez des offres personnalisées, des promotions ou des recommandations à des segments spécifiques de clients, augmentant ainsi les chances de conversion et de satisfaction client.
    \item \textbf{Fidélisation de la clientèle}: Utilisez la solution de smart data d'Izicap pour identifier les clients qui risquent de résilier leur abonnement ou de ne plus acheter chez vous. Développez des stratégies de fidélisation en leur offrant des incitations, des programmes de fidélité ou des communications personnalisées afin de les maintenir engagés et fidèles à votre marque.
    \item \textbf{Suivi des performances}: Surveillez régulièrement les performances de vos campagnes marketing et de vos efforts d'engagement client. La solution d'Izicap peut vous fournir des métriques et des rapports pour évaluer l'efficacité de vos stratégies et apporter des ajustements basés sur les données lorsque nécessaire.
    \item \textbf{Amélioration continue}: Les solutions de smart data sont les plus efficaces lorsqu'elles sont utilisées de manière itérative. Analysez régulièrement les données, adaptez vos stratégies et peaufinez votre approche en fonction de nouvelles informations et de l'évolution du comportement client.
\end{enumerate}


\begin{figure}[H]
\centering
\includegraphics[width=0.6\linewidth]{images/smart-data-izicap.png}
\caption{Smart data Izicap}\label{fig:smart-data-Izicap}
\end{figure}


\section{L'architecture du système -Smart Data-}

La figure ci-dessous visualise la structure et les composants clés de notre système, ainsi que les interactions entre eux, elle constitue ainsi le fondement de notre système et joue un rôle essentiel dans la fourniture de fonctionnalités et de services à nos utilisateurs.

\begin{figure}[H]
\centering
\includegraphics[width=\linewidth]{images/archi-globale.png}
\caption{Architecture du système actuelle}\label{fig:architecture-monolithique}
\end{figure}

Ce schéma représente le fonctionnement de l'architecture monolithique actuelle qui présente plusieurs inconvénients qui peuvent entraver l'efficacité, la flexibilité et la maintenabilité du système. Voici une description détaillée des principaux inconvénients:

\begin{enumerate}
    \item \textbf{Complexité et dépendances}: L'architecture monolithique implique que tous les modules et fonctionnalités du système sont regroupés en un seul bloc. Cela crée une forte interdépendance entre les différents composants, rendant la compréhension, la gestion et les mises à jour complexes. Les modifications apportées à une partie du système peuvent avoir des répercussions sur d'autres parties, ce qui rend les tests, les déploiements et les corrections d'erreurs plus difficiles.
    \item \textbf{Scalabilité limitée}: L'architecture monolithique a souvent des difficultés à s'adapter à une augmentation de la charge ou à une demande croissante. Étant donné que tous les composants sont regroupés dans un seul monolithe, il est difficile de faire évoluer sélectivement une partie spécifique du système. Cela peut entraîner des goulots d'étranglement, des performances réduites et une mauvaise répartition des ressources lorsqu'il s'agit de traiter des volumes de données importants ou de gérer une augmentation du nombre d'utilisateurs.
    \item \textbf{Déploiements complexes et risques élevés}: Avec une architecture monolithique, les déploiements nécessitent la mise à jour de l'ensemble du système, même pour des modifications mineures. Cela augmente les risques d'erreurs et de régressions, car une seule erreur peut entraîner l'indisponibilité du système tout entier. De plus, les déploiements doivent être soigneusement planifiés et coordonnés, ce qui peut entraîner des temps d'arrêt plus longs et des interruptions de service pour les utilisateurs.
    \item \textbf{Difficulté de choix technologiques}: Dans une architecture monolithique, les technologies utilisées sont souvent liées et intégrées de manière étroite. Cela peut rendre difficile l'adoption de nouvelles technologies ou l'intégration de composants spécialisés. Les mises à niveau de versions ou les ajouts de nouvelles fonctionnalités peuvent être limités par les choix technologiques initiaux, ce qui peut entraver l'innovation et l'adaptation aux évolutions du marché.
    \item \textbf{Cohésion et responsabilités}: L'architecture monolithique ne facilite pas une séparation claire des responsabilités et des fonctionnalités. Les différents modules du système sont souvent intimement liés et peuvent partager des fonctionnalités communes. Cela rend difficile l'isolation des problèmes, la maintenance spécifique des modules et la réutilisation de code spécifique à un domaine.
\end{enumerate}

\section{Problématique et besoin fonctionnel}
La problématique qui se pose réside dans l'architecture monolithique actuellement utilisée, laquelle présente des limitations en termes de scalabilité, de flexibilité et de maintenabilité. Les contraintes imposées par cette architecture rendent complexe le déploiement de nouvelles fonctionnalités et entraînent des perturbations potentielles dans l'ensemble du système. De plus, l'utilisation d'une base de données relationnelle traditionnelle, telle que MariaDB, se révèle insuffisante pour traiter efficacement de grands volumes de données transactionnelles, il convient de souligner que l'application Smart Data existe depuis maintenant 10 ans et a été développée en utilisant le langage Groovy. Cette longue période d'existence témoigne de la stabilité et de la maturité de notre système. Cependant, l'utilisation du langage Groovy peut présenter certaines contraintes en termes de maintenabilité et d'évolutivité, notamment lorsqu'il s'agit d'intégrer de nouvelles technologies et de gérer des architectures plus modernes.
 
\subsection*{Besoin fonctionnel:}

Afin de répondre à ces problématiques, différents besoins fonctionnels ont été identifiés :

\begin{enumerate}
    \item[$\bullet$] \textbf{Scalabilité}: Il est nécessaire de disposer d'une architecture qui puisse s'adapter facilement à une croissance du nombre d'utilisateurs, des transactions et des données. Il est essentiel que notre système puisse évoluer harmonieusement et maintenir ses performances, même face à une augmentation significative de la charge de travail.
    \item[$\bullet$] \textbf{Flexibilité}: Nous devons être en mesure d'introduire de nouvelles fonctionnalités de manière indépendante et de les déployer sans perturber l'ensemble du système. Une approche basée sur des microservices et des microfrontends nous permettra d'atteindre cette flexibilité, en facilitant le développement, les tests et le déploiement isolé de chaque composant.
    \item[$\bullet$] \textbf{Performances améliorées}: Il est primordial de disposer d'une infrastructure de données capable de gérer efficacement d'importants volumes de données transactionnelles. En remplaçant MariaDB par Delta Lake, nous pourrons tirer parti de fonctionnalités avancées telles que la gestion des transactions ACID et la compatibilité avec des outils d'analyse performants, ce qui améliorera sensiblement les performances et la fiabilité de notre système.
    \item[$\bullet$] \textbf{Séparation des responsabilités}: Nous visons une meilleure séparation des responsabilités entre les différents composants de notre système. Les microservices nous permettront de découpler les fonctionnalités, de les attribuer à des équipes spécifiques et de favoriser une gestion du code plus efficace, une maintenance simplifiée et une évolutivité accrue.
\end{enumerate}


\section{Objectives du stages}
Le projet s'inscrit dans le cadre du renouvellement des fonctionnalités de l'application Smart Data, visant à améliorer sa scalabilité, sa flexibilité et ses performances. Dans ce contexte, les objectifs du stage sont les suivants :

\begin{enumerate}
    \item Étudier et analyser l'architecture actuelle de l'application SMART DATA afin de comprendre les contraintes et les limitations qui entravent sa croissance et son évolution.
    \item Proposer et concevoir une stratégie de migration de l'architecture monolithique vers une architecture basée sur des microservices et des microfrontends, permettant ainsi une meilleure modularité et une plus grande flexibilité dans le développement et le déploiement des fonctionnalités.
    \item Mettre en œuvre multiples microservices avec Springboot dans le cadre de la nouvelle architecture, en utilisant une technologie appropriée et en veillant à son intégration harmonieuse avec les autres composants du système.
    \item Évaluer les avantages et les implications de l'utilisation de Delta Lake en remplacement de la base de données MariaDB, en mettant l'accent sur les performances, la gestion des transactions et l'intégration avec les outils d'analyse.
    \item Intégrer Trino dans l'architecture pour permettre l'exécution de requêtes SQL complexes et optimiser le traitement des données, en assurant une collaboration efficace avec l'équipe de développement.
    \item Mettre en place une infrastructure de déploiement et de gestion des microservices, en utilisant des outils tels que Kubernetes, Docker et Jenkins, pour faciliter le déploiement, la mise à l'échelle et la gestion des composants.
    \item Concevoir et mettre en œuvre des tests unitaires et des tests d'intégration pour assurer la qualité et la fiabilité des nouveaux composants développés dans le cadre de l'architecture basée sur les microservices.
    \item Documenter de manière approfondie le processus de migration et les choix technologiques effectués, fournissant des instructions claires pour la maintenance future de l'architecture et la gestion des mises à jour.
    \item Collaborer étroitement avec l'équipe existante pour faciliter la transition vers la nouvelle architecture, en offrant un soutien technique, des conseils et des formations sur les nouvelles technologies utilisées.
\end{enumerate}

\section{Processus de réalisation du projet}

Le processus de réalisation du projet s'est déroulé en plusieurs itérations appelées `sprints', d'une durée généralement fixe de deux semaines. Chaque sprint était axé sur la livraison d'incréments fonctionnels de l'application, permettant ainsi d'obtenir rapidement des résultats tangibles.

Voici les étapes clés du processus de réalisation du projet, basé sur la méthodologie Scrum:

\begin{enumerate}
    \item \textbf{Définition du backlog du produit:} En collaboration avec les parties prenantes et l'équipe de développement, nous avons identifié et priorisé les fonctionnalités à développer et à intégrer dans l'architecture basée sur les microservices.
    \item \textbf{Planification du sprint:} Au début de chaque sprint, nous avons organisé une réunion de planification pour définir les objectifs spécifiques du sprint, sélectionner les tâches à réaliser et estimer les efforts nécessaires.
    \item \textbf{Développement itératif:} L'équipe de développement a travaillé de manière itérative sur les tâches assignées, en se concentrant sur la réalisation des fonctionnalités identifiées pour le sprint en cours.
    \item \textbf{Réunions quotidiennes de stand up:} Chaque jour, l'équipe s'est réunie pour une brève réunion de stand-up afin de partager les progrès, les obstacles éventuels et coordonner les activités.
    \item \textbf{Revue de sprint:} À la fin de chaque sprint, nous avons organisé une revue de sprint pour présenter les fonctionnalités développées et obtenir des retours des parties prenantes. Cela nous a permis de valider les résultats obtenus et de planifier les prochaines étapes.
    \item \textbf{Rétrospective de sprint:} Après la revue de sprint, nous avons mené une rétrospective pour évaluer le déroulement du sprint, identifier les points forts et les points à améliorer, et ajuster notre approche en conséquence.
    \item \textbf{Itérations suivantes:} Le processus de planification, de développement itératif, de revue de sprint et de rétrospective s'est répété pour chaque sprint suivant, permettant ainsi une progression incrémentale vers les objectifs du projet.
\end{enumerate}

\begin{figure}[H]
\centering
\includegraphics[width=\linewidth]{images/scrum.png}
\caption{Méthodologie Scrum}\label{fig:scrum}
\end{figure}

\section{Planification du projet}
Pour la planification du projet, nous avons utilisé la méthode de modélisation du réseau de dépendance entre les tâches. Cette approche nous a permis de décomposer le travail en différentes tâches structurées et d'établir des relations de dépendance entre elles.

Nous avons également utilisé la technique du diagramme de Gantt pour représenter graphiquement les tâches et les ressources du projet dans le temps. En ligne, nous avons listé les différentes tâches, et en colonne, nous avons défini les jours, les semaines ou les mois. Chaque tâche a été représentée par une barre dont la longueur est proportionnelle à la durée estimée de cette tâche.

Le diagramme de Gantt nous a permis de visualiser la répartition des tâches, leur durée et leur succession. Certaines tâches se sont réalisées en séquence, tandis que d'autres ont pu être réalisées en parallèle, de manière partielle ou totale. Cette représentation claire et visuelle nous a aidés à planifier le projet en déterminant et en organisant les différentes tâches de manière à assurer une gestion efficace du projet.

La figure suivante présente le diagramme de Gantt détaillant la planification du projet, avec les tâches ordonnées dans le temps et leur durée respective. Cela nous a permis de suivre l'avancement du projet, d'identifier les éventuels retards et de prendre les mesures appropriées pour les résoudre.

\begin{figure}[H]
\centering
\includegraphics[width=\linewidth]{images/gantt.png}
\caption{Diagramme de Gantt}\label{fig:gantt}
\end{figure}

\addcontentsline{toc}{section}{Conclusion}
\section*{Conclusion}

En conclusion de cette section, nous avons présenté la planification du projet de renouvellement des fonctionnalités de l'application SMART DATA. Nous avons élaboré un planning basé sur la modélisation du réseau de dépendance entre les tâches, en utilisant la technique du diagramme de Gantt. Ce diagramme nous a permis de visualiser les différentes activités, leur durée estimée et les ressources nécessaires.

Au cours de la planification, nous avons identifié les principales activités du projet, telles que l'analyse de l'architecture actuelle, le découpage de l'architecture monolithique, l'implémentation des microservices, l'intégration des microfrontends, la mise en place de Delta Lake, l'évaluation des connecteurs, la configuration et le déploiement de Trino, les tests et la validation des fonctionnalités, la documentation du projet, la préparation du déploiement, ainsi que la formation et la sensibilisation des utilisateurs.

Nous avons également pris en compte la durée totale du projet, qui est estimée à 5 mois, et avons ajusté les durées des activités en conséquence. Cela nous a permis d'avoir une vision plus précise de la planification temporelle du projet.
    
    \chapter{Cheminement de la solution}\label{chap:2}
    \addcontentsline{toc}{section}{Introduction}
\section*{Introduction}

Dans ce chapitre, nous discuterons du cheminement de la solution proposée pour répondre aux besoins identifiés. Nous aborderons les différentes composantes de la solution, à savoir la partie data, le backend et le frontend.

\section{Côté Data}
Dans le cadre de notre solution, nous proposons de remplacer l'exporter worker existant et les bases de données MariaDB par l'utilisation de Delta Lake sur le Cloud Object Store. Delta Lake est un système de gestion de données open-source qui offre des fonctionnalités avancées telles que la gestion des transactions, la réplication des données et la prise en charge de schémas évolutifs. En utilisant Delta Lake, nous pouvons garantir la fiabilité et la cohérence des données tout en bénéficiant des avantages d'un stockage dans le cloud.

\section{Côté Backend}
Pour la mise en œuvre des microservices, nous proposons d'utiliser Spring Boot, un framework Java populaire pour le développement d'applications. En utilisant Spring Boot, nous pouvons créer des microservices autonomes, indépendants les uns des autres, qui peuvent être développés, déployés et scalés individuellement. Spring Boot fournit également des fonctionnalités telles que la gestion de la persistance des données, la sécurité et la création d'API REST, ce qui facilite le développement des microservices.

\begin{figure}[H]
\centering
\includegraphics[width=\linewidth]{images/delta-lake-microservices.png}
\caption{Delta lake connectés à des microservices}\label{fig:schema-delta}
\end{figure}

\section{Côté Frontend}
Pour la mise en œuvre des microservices, nous avons décider d'utiliser Spring Boot, un framework Java populaire pour le développement d'applications. En utilisant Spring Boot, nous pouvons créer des microservices autonomes, indépendants les uns des autres, qui peuvent être développés, déployés et scalés individuellement. Spring Boot fournit également des fonctionnalités telles que la gestion de la persistance des données, la sécurité et la création d'API REST, ce qui facilite le développement des microservices.

\begin{figure}[H]
\centering
\includegraphics[width=\linewidth]{images/microfrontends.png}
\caption{Schéma des microfrontends}\label{fig:schema-microfrontends}
\end{figure}

\section*{Conclusion}
\addcontentsline{toc}{section}{Conclusion}

En adoptant ce cheminement de solution, nous visons à moderniser et à améliorer l'architecture de notre système. En remplaçant l'exporter worker et les bases de données MariaDB par Delta Lake sur le Cloud Object Store, nous assurons une meilleure gestion des données, scalable et moins chère. En utilisant des microservices avec Spring Boot, nous obtenons une architecture plus modulaire, évolutive et facile à maintenir du côté backend. Enfin, en utilisant des micro-frontends avec React, nous améliorons la flexibilité et la maintenabilité de l'interface utilisateur.
    
    \chapter{Implementation de la solution}\label{chap:3}
    \addcontentsline{toc}{section}{Introduction}

\section*{Introduction}

\section{Définition}
Trino est un moteur de requête SQL distribué open source. Il s'agit d'un hard fork du projet Presto original créé par Facebook. Il permet aux développeurs d'exécuter des analyses interactives sur de gros volumes de données. Avec Trino, les organisations peuvent facilement utiliser leurs compétences SQL existantes pour interroger des données sans avoir à apprendre de nouveaux langages complexes. L'architecture est assez similaire aux systèmes traditionnels de traitement analytique en ligne (OLAP) utilisant des architectures informatiques distribuées, dans lesquelles un nœud de contrôleur coordonne plusieurs nœuds de travail.
\section{Comment ça fonctionne}
Trino est un système distribué qui s'exécute sur Hadoop et utilise une architecture similaire aux bases de données de traitement massivement parallèle (MPP). Il a un nœud coordinateur travaillant avec plusieurs nœuds de travail. Les utilisateurs soumettent SQL au coordinateur qui utilise le moteur de requête et d'exécution pour analyser, planifier et planifier un plan de requête distribué sur les nœuds de travail. Il prend en charge le SQL ANSI standard, y compris les requêtes complexes, les agrégations de jointures et les jointures externes.

Tirant parti de cette architecture, le moteur de requête Trino est capable de traiter des requêtes SQL sur de grandes quantités de données en parallèle sur un cluster d'ordinateurs ou de nœuds. Trino s'exécute en tant que processus à serveur unique sur chaque nœud. Plusieurs nœuds exécutant Trino, qui sont configurés pour collaborer les uns avec les autres, constituent un cluster Trino.

La figure suivante affiche une vue d'ensemble de haut niveau d'un cluster Trino composé d'un coordinateur et de plusieurs nœuds de travail. Un utilisateur Trino se connecte au coordinateur avec un client, tel qu'un outil utilisant le pilote JDBC ou la CLI Trino. Le coordinateur collabore ensuite avec les workers, qui accèdent aux sources de données.

\begin{figure}[htbp]
\centering
\includegraphics[width=\linewidth]{images/trino_architecture.png}
\caption{Vue d'ensemble de l'architecture Trino avec le coordinateur et les workers}\label{fig:trino-architecture}
\end{figure}

\begin{enumerate}
	\item Un coordinateur est un serveur Trino qui gère les requêtes entrantes et gère les workers pour exécuter les requêtes.
	\item Un worker est un serveur Trino responsable de l'exécution des tâches et du traitement des données.
	\item Le service de découverte s'exécute généralement sur le coordinateur et permet aux workers de s'inscrire pour participer au cluster.
	\item Toutes les communications et tous les transferts de données entre les clients, le coordinateur et les workers utilisent des interactions basées sur REST sur HTTP/HTTPS.
\end{enumerate}

La figure suivante montre comment la communication au sein du cluster se produit entre le coordinateur et les workers, ainsi que d'un worker à l'autre. Le coordinateur discute avec les workers pour attribuer le travail, mettre à jour le statut et récupérer l'ensemble de résultats de niveau supérieur à renvoyer aux utilisateurs. Les workers se parlent pour récupérer des données à partir de tâches en amont, exécutées sur d'autres workers. Et les workers récupèrent les ensembles de résultats à partir de la source de données.

\begin{figure}[htbp]
\centering
\includegraphics[width=\linewidth]{images/trino_communication.png}
\caption{Communication entre le coordinateur et les workers dans un cluster Trino}\label{fig:trino-communication}
\end{figure}

\section{Coordinateur}
Le coordinateur Trino est le serveur responsable de la réception des instructions SQL des utilisateurs, de l'analyse de ces instructions, de la planification des requêtes et de la gestion des nœuds de travail. C'est le cerveau d'une installation Trino et le nœud auquel un client se connecte. Les utilisateurs interagissent avec le coordinateur via la CLI Trino, les applications utilisant les pilotes JDBC ou ODBC, ou toute autre bibliothèque client disponible pour une variété de langues. Le coordinateur accepte les instructions SQL du client telles que les requêtes SELECT pour l'exécution.

Chaque installation Trino doit avoir un coordinateur aux côtés d'un ou plusieurs workers. À des fins de développement ou de test, une seule instance de Trino peut être configurée pour remplir les deux rôles.

Le coordinateur suit l'activité de chaque worker et coordonne l'exécution d'une requête. Le coordinateur crée un modèle logique d'une requête impliquant une série d'étapes.

Une fois qu'il reçoit une instruction SQL, le coordinateur est responsable de l'analyse, de l'analyse, de la planification et de la planification de l'exécution de la requête sur les nœuds de travail Trino. L'instruction est traduite en une série de tâches connectées s'exécutant sur un cluster de workers. Au fur et à mesure que les workers traitent les données, les résultats sont récupérés par le coordinateur et exposés aux clients sur un tampon de sortie. Une fois qu'un tampon de sortie est complètement lu par le client, le coordinateur demande plus de données aux workers au nom du client. Les workers, à leur tour, interagissent avec les sources de données pour en obtenir les données. En conséquence, les données sont continuellement demandées par le client et fournies par les workers à partir de la source de données jusqu'à ce que l'exécution de la requête soit terminée.

Les coordinateurs communiquent avec les workers et les clients à l'aide d'un protocole basé sur HTTP.
\begin{figure}[htbp]
\centering
\includegraphics[width=\linewidth]{images/trino_communication_processing.png}
\caption{Communication client, coordinateur et worker traitant une instruction SQL}\label{fig:trino-communication-precessing}
\end{figure}

\section{Workers}
Un worker Trino est un serveur dans une installation Trino. Il est responsable de l'exécution des tâches assignées par le coordinateur et du traitement des données. Les nœuds de travail récupèrent des données à partir de sources de données à l'aide de connecteurs, puis échangent des données intermédiaires entre eux. Les données finales qui en résultent sont transmises au coordinateur. Le coordonnateur est chargé de recueillir les résultats des workers et de fournir les résultats finaux au client.

Lors de l'installation, les agents sont configurés pour connaître le nom d'hôte ou l'adresse IP du service de découverte du cluster. Lorsqu'un agent démarre, il s'annonce au service de découverte, qui le met à la disposition du coordinateur pour l'exécution de la tâche.

Les workers communiquent avec d'autres workers et le coordinateur à l'aide d'un protocole basé sur HTTP.

La figure suivante montre comment plusieurs workers récupèrent des données à partir des sources de données et collaborent pour traiter les données, jusqu'à ce qu'un worker puisse fournir les données au coordinateur.

\begin{figure}[htbp]
\centering
\includegraphics[width=\linewidth]{images/trino_workers.png}
\caption{Les workers d'un cluster collaborent pour traiter les instructions et les données SQL}\label{fig:trino-workers}
\end{figure}

\section{Architecture basée sur les connecteurs}
Au cœur de la séparation du stockage et du calcul dans Trino se trouve l'architecture basée sur les connecteurs. Un connecteur fournit à Trino une interface pour accéder à une source de données arbitraire.

Chaque connecteur fournit une abstraction basée sur une table sur la source de données sous-jacente. Tant que les données peuvent être exprimées en termes de tables, de colonnes et de lignes à l'aide des types de données disponibles pour Trino, un connecteur peut être créé et le moteur de requête peut utiliser les données pour le traitement des requêtes.

Trino fournit une interface de fournisseur de services (SPI), qui est un type d'API utilisé pour implémenter un connecteur. En implémentant le SPI dans un connecteur, Trino peut utiliser des opérations standard en interne pour se connecter à n'importe quelle source de données et effectuer des opérations sur n'importe quelle source de données. Le connecteur prend en charge les détails relatifs à la source de données spécifique.

\begin{enumerate}
	\item[$\bullet$] Opérations pour récupérer les métadonnées de table/vue/schéma
	\item[$\bullet$] Opérations pour produire des unités logiques de partitionnement de données, afin que Trino puisse paralléliser les lectures et les écritures
	\item[$\bullet$] Sources et récepteurs de données qui convertissent les données source vers/depuis le format en mémoire attendu par le moteur de requête 
\end{enumerate}

Trino fournit de nombreux connecteurs aux systèmes, vous trouverez la liste des connecteurs au moment de la rédaction de ce rapport ci-dessous.

% chktex-file 44
\begin{table}[ht]
\centering
	\begin{tabular}{|c|c|}
	\hline
	\textbf{Nom de connecteur} & \textbf{Lien de documentation} \\ \hline
	Accumulo & \href{https://trino.io/docs/current/connector/accumulo.html}{https://trino.io/docs/current/connector/accumulo.html} \\ \hline
	Atop & \href{https://trino.io/docs/current/connector/atop.html}{https://trino.io/docs/current/connector/atop.html} \\ \hline
	BigQuery & \href{https://trino.io/docs/current/connector/bigquery.html}{https://trino.io/docs/current/connector/bigquery.html} \\ \hline
	Black Hole & \href{https://trino.io/docs/current/connector/blackhole.html}{https://trino.io/docs/current/connector/blackhole.html} \\ \hline
	Cassandra & \href{https://trino.io/docs/current/connector/cassandra.html}{https://trino.io/docs/current/connector/cassandra.html} \\ \hline
	ClickHouse & \href{https://trino.io/docs/current/connector/clickhouse.html}{https://trino.io/docs/current/connector/clickhouse.html} \\ \hline
	Delta Lake & \href{https://trino.io/docs/current/connector/delta-lake.html}{https://trino.io/docs/current/connector/delta-lake.html} \\ \hline
	Druid & \href{https://trino.io/docs/current/connector/druid.html}{https://trino.io/docs/current/connector/druid.html} \\ \hline
	Elasticsearch & \href{https://trino.io/docs/current/connector/elasticsearch.html}{https://trino.io/docs/current/connector/elasticsearch.html} \\ \hline
	Google Sheets & \href{https://trino.io/docs/current/connector/googlesheets.html}{https://trino.io/docs/current/connector/googlesheets.html} \\ \hline
	Hive & \href{https://trino.io/docs/current/connector/hive.html}{https://trino.io/docs/current/connector/hive.html} \\ \hline
	Hudi & \href{https://trino.io/docs/current/connector/hudi.html}{https://trino.io/docs/current/connector/hudi.html} \\ \hline
	Iceberg & \href{https://trino.io/docs/current/connector/iceberg.html}{https://trino.io/docs/current/connector/iceberg.html} \\ \hline
	Ignite & \href{https://trino.io/docs/current/connector/ignite.html}{https://trino.io/docs/current/connector/ignite.html} \\ \hline
	JMX & \href{https://trino.io/docs/current/connector/jmx.html}{https://trino.io/docs/current/connector/jmx.html} \\ \hline
	Kafka & \href{https://trino.io/docs/current/connector/kafka.html}{https://trino.io/docs/current/connector/kafka.html} \\ \hline
	Kinesis & \href{https://trino.io/docs/current/connector/kinesis.html}{https://trino.io/docs/current/connector/kinesis.html} \\ \hline
	Kudu & \href{https://trino.io/docs/current/connector/kudu.html}{https://trino.io/docs/current/connector/kudu.html} \\ \hline
	Local File & \href{https://trino.io/docs/current/connector/localfile.html}{https://trino.io/docs/current/connector/localfile.html} \\ \hline
	MariaDB & \href{https://trino.io/docs/current/connector/mariadb.html}{https://trino.io/docs/current/connector/mariadb.html} \\ \hline
	Memory & \href{https://trino.io/docs/current/connector/memory.html}{https://trino.io/docs/current/connector/memory.html} \\ \hline
	MongoDB & \href{https://trino.io/docs/current/connector/mongodb.html}{https://trino.io/docs/current/connector/mongodb.html} \\ \hline
	MySQL & \href{https://trino.io/docs/current/connector/mysql.html}{https://trino.io/docs/current/connector/mysql.html} \\ \hline
	Oracle & \href{https://trino.io/docs/current/connector/oracle.html}{https://trino.io/docs/current/connector/oracle.html} \\ \hline
	Phoenix & \href{https://trino.io/docs/current/connector/phoenix.html}{https://trino.io/docs/current/connector/phoenix.html} \\ \hline
	Pinot & \href{https://trino.io/docs/current/connector/pinot.html}{https://trino.io/docs/current/connector/pinot.html} \\ \hline
	PostgreSQL & \href{https://trino.io/docs/current/connector/postgresql.html}{https://trino.io/docs/current/connector/postgresql.html} \\ \hline
	Prometheus & \href{https://trino.io/docs/current/connector/prometheus.html}{https://trino.io/docs/current/connector/prometheus.html} \\ \hline
	Redis & \href{https://trino.io/docs/current/connector/redis.html}{https://trino.io/docs/current/connector/redis.html} \\ \hline
	Redshift & \href{https://trino.io/docs/current/connector/redshift.html}{https://trino.io/docs/current/connector/redshift.html} \\ \hline
	SingleStore & \href{https://trino.io/docs/current/connector/singlestore.html}{https://trino.io/docs/current/connector/singlestore.html} \\ \hline
	SQL Server & \href{https://trino.io/docs/current/connector/sqlserver.html}{https://trino.io/docs/current/connector/sqlserver.html} \\ \hline
	System & \href{https://trino.io/docs/current/connector/system.html}{https://trino.io/docs/current/connector/system.html} \\ \hline
	Thrift & \href{https://trino.io/docs/current/connector/thrift.html}{https://trino.io/docs/current/connector/thrift.html} \\ \hline
	TPCDS & \href{https://trino.io/docs/current/connector/tpcds.html}{https://trino.io/docs/current/connector/tpcds.html} \\ \hline
	TPCH & \href{https://trino.io/docs/current/connector/tpch.html}{https://trino.io/docs/current/connector/tpch.html} \\ \hline
\end{tabular}
\caption{Liste des connecteurs Trino et leur documentation}
\end{table}

Le SPI de Trino vous donne également la possibilité de créer vos propres connecteurs personnalisés. Cela peut être nécessaire si vous avez besoin d'accéder à une source de données sans connecteur compatible. Si vous finissez par créer un connecteur, nous vous encourageons fortement à en savoir plus sur la communauté open source Trino, à utiliser notre aide et à contribuer votre connecteur. Consultez « Ressources Trino » pour plus d'informations. Un connecteur personnalisé peut également être nécessaire si vous disposez d'une source de données unique ou propriétaire au sein de votre organisation. C'est ce qui permet aux utilisateurs de Trino d'interroger n'importe quelle source de données en utilisant SQL, vraiment SQL-on-Anything.

La figure suivante montre comment le Trino SPI inclut des interfaces distinctes pour les métadonnées, les statistiques de données et l'emplacement des données utilisées par le coordinateur, et pour le flux de données utilisé par les workers.

\begin{figure}[htbp]
\centering
\includegraphics[width=\linewidth]{images/trino_connector.png}
\caption{Vue d'ensemble de l'interface du fournisseur de services Trino (SPI)}\label{fig:trino-connector}
\end{figure}

Les connecteurs Trino sont des plug-ins chargés par chaque serveur au démarrage. Ils sont configurés par des paramètres spécifiques dans les fichiers de propriétés du catalogue et chargés à partir du répertoire des plug-ins.

\section{Catalogues, Schémas et Tables}
Le cluster Trino traite toutes les requêtes en utilisant l'architecture basée sur les connecteurs décrite précédemment. Chaque configuration de catalogue utilise un connecteur pour accéder à une source de données spécifique. La source de données expose un ou plusieurs schémas dans le catalogue. Chaque schéma contient des tables qui fournissent les données dans des lignes de table avec des colonnes utilisant différents types de données. Vous pouvez en savoir plus sur les catalogues, les schémas, les tables. Plus précisément dans `Catalogues', `Schémas' et `Tables'.

\addcontentsline{toc}{section}{Conclusion}
\section*{Conclusion}
L'architecture Trino a un coordinateur recevant les demandes des utilisateurs, puis utilisant des travailleurs pour assembler toutes les données à partir des sources de données. Chaque requête est traduite en un plan de requête distribué de tâches en plusieurs étapes. Les données sont renvoyées par les connecteurs en fractions et traitées en plusieurs étapes jusqu'à ce que le résultat final soit disponible et fourni à l'utilisateur par le coordinateur.

    \chapter{Technologies Utilisées}\label{chap:4}
    \addcontentsline{toc}{section}{Introduction}

\section*{Introduction}


\section{}


\section*{Conclusion}

\else
    \chapter{Izicap}\label{chap:1}
    \minitoc
\addcontentsline{toc}{section}{Introduction}
\section*{Introduction}

\section{About}
Izicap is pioneering the use of payment card data and turning it into powerful customer knowledge & business
insights for merchants, allowing them to run their own loyalty programs and digital marketing campaigns. These
marketing services, provided by acquirers using Izicap’s SaaS solutions, quickly restore growth to merchant
businesses by building their customers (the card-holders) spending and stickiness. Izicap’s innovative card-linked
CRM & Loyalty solution gives acquirers a competitive edge by monetising their payment transactions data,
generating new sources of income and improving their retention capabilities. After having solidly established itself in
France thanks to partnerships with Groupe BPCE and Crédit Agricole, Izicap partnered with Nexi, the leading
acquirer & Fintech in Italy and has joined Mastercard’s StartPath program with an aim to dramatically expand its
worldwide reach. Izicap partners with leading payment solution providers such as Ingenico, Verifone, Poynt and PAX,
and makes its card-linked CRM & Loyalty solution available on the most popular and innovative payments terminals

\section{Organizational chart}
Information on Izicap

\addcontentsline{toc}{section}{Conclusion}
\section*{Conclusion}

    
    \chapter{Delta Lake}\label{chap:2}
    \addcontentsline{toc}{section}{Introduction}
\section*{Introduction}

Data warehouses and data lakes are the most common central data repositories employed by most data-driven organizations today, each with its own strengths and tradeoffs. For one, while data warehouses allow businesses to organize historical datasets for use in business intelligence (BI) and analytics, they quickly become more cost-intensive as datasets grow because of the combined use of compute and storage resources. Additionally, data warehouses can’t handle the varied nature of data (structured, unstructured, and semi-structured) seen today.

In this chapter, we will explore the key features of Delta Lake, how it works, and why it is a good choice for big data processing. We will also provide examples of how to use Delta Lake with other big data tools, such as Spark and Trino later on in this report.
\section{Definition}
Delta Lake is an open-source storage layer built atop a data lake that confers reliability and ACID (Atomicity, Consistency, Isolation, and Durability) transactions. It enables a continuous and simplified data architecture for organizations. A data lake stores data in Parquet formats and enables a lakehouse data architecture, which helps organizations achieve a single, continuous data system that combines the best features of both the data warehouse and data lake while supporting streaming and batch processing.

\section{How Delta Lake Works}
A Delta Lake enables the building of a data lakehouse. Common lakehouses include the Databricks Lakehouse and Azure Databricks. This continuous data architecture allows organizations to harness the benefits of data warehouses and data lakes with reduced management complexity and cost. Here are some ways Delta Lake improves the use of data warehouses and lakes:
\begin{itemize}
    \item \textbf{Enables a lakehouse architecture:} Delta Lake enables a continuous and simplified data architecture that allows organizations to handle and process massive volumes of streaming and batch data without the management and operational hassles involved in managing streaming, data warehouses, and data lakes separately.
    \item \textbf{Enables intelligent data management for data lakes:} Delta Lake offers efficient and scalable metadata handling, which provides information about the massive data volumes in data lakes. With this information, data governance and management tasks proceed more efficiently.
    \item \textbf{Schema enforcement for improved data quality:} Because data lakes lack a defined schema, it becomes easy for bad/incompatible data to enter data systems. There is improved data quality thanks to automatic schema validation, which validates DataFrame and table compatibility before writes.
    \item \textbf{Enables ACID transactions:} Most organizational data architectures involve a lot of ETL and ELT movement in and out of data storage, which opens it up to more complexity and failure at node entry points. Delta Lake ensures the durability and persistence of data during ETL and other data operations. Delta lake captures all changes made to data during data operations in a transaction log, thereby ensuring data integrity and reliability during data operations.
\end{itemize}

\section{Delta Lake Architecture Diagram}
\begin{flushleft}
Delta Lake is an improvement from the lambda architecture whereby streaming and batch processing occur parallel, and results merge to provide a query response. However, this method means more complexity and difficulty maintaining and operating both the streaming and batch processes. Unlike the lambda architecture, Delta Lake is a continuous data architecture that combines streaming and batch workflows in a shared file store through a connected pipeline.
\end{flushleft}

The stored data file has three layers, with the data getting more refined as it progresses downstream in the dataflow:

\begin{enumerate}
\item[$\bullet$] \textbf{Bronze tables:} This table contains the raw data ingested from multiple sources like the Internet of Things (IoT) systems, CRM, RDBMS, and JSON files.
\item[$\bullet$] \textbf{Silver tables:} This layer contains a more refined view of our data after undergoing transformation and feature engineering processes.
\item[$\bullet$] \textbf{Gold tables:} This final layer is often made available for end users in BI reporting and analysis or use in machine learning processes.
\end{enumerate}

\begin{figure}[htbp]
\centering
\includegraphics[width=\linewidth]{images/delta_lake_architecture.png}
\caption{Delta Lake multi-hop architecture}\label{fig:delta-lake-architecture}
\end{figure}

\section{Key benefits and features of Delta Lake}
\begin{itemize}
    \item[\textbullet] \textbf{Audit trails and history:} In Delta Lake, every write exists as a transaction and is serially recorded in a transaction log. Therefore, any changes or commits made to the transaction log are recorded, leaving a complete trail for use in historical audits, versioning, or for time traveling purposes. This Delta Lake feature helps ensure data integrity and reliability for business data operations.
    \item[\textbullet] \textbf{Time traveling and data versioning:} Because each write creates a new version and stores the older version in the transaction log, users can view/revert to older data versions by providing the timestamp or version number of an existing table or directory to the Sparks read API\@. Using the version number provided, the Delta Lake then constructs a full snapshot of the version with the information provided by the transaction log. Rollbacks and versioning play a vital role in machine learning experimentation, whereby data scientists iteratively change hyperparameters to train models and can revert to changes if needed.
    \item[\textbullet] \textbf{Unifies batch and stream processing:} Every table in a Delta Lake is a batch and streaming sink. With Sparks structured streaming, organizations can efficiently stream and process streaming data. Additionally, with the efficient metadata handling, ease of scale, and ACID quality of each transaction, near-real-time analytics become possible without utilizing a more complicated two-tiered data architecture.
    \item[\textbullet] \textbf{Efficient and scalable metadata handling:} Delta Lakes store metadata information in the transaction log and leverages Spark's distributed processing power to quickly process and efficiently read and handle large volumes of data metadata, thus improving data governance.
    \item[\textbullet] \textbf{ACID transactions:} Delta Lakes ensure that users always see a consistent data view in a table or directory. It guarantees this by capturing every change made in a transaction log and isolating it at the strongest isolation level, the serializable level. In the serializable level, every existing operation has and follows a serial sequence that, when executed one by one, provides the same result as seen in the table.
    \item[\textbullet] \textbf{Data Manipulation Language operations:} Delta Lakes supports DML operations like updates, deletes, and merges, which play a role in complex data operations like change-data-capture (CDC), streaming upserts, and slowly-changing-dimension (SCD). Operations like CDC ensure data synchronization in all data systems and minimizes the time and resources spent on ELT operations. For instance, using the CDC, instead of ETL-ing all the available data, only the recently updated data since the last operation undergoes a transformation.
    \item[\textbullet] \textbf{Schema enforcement:} Delta Lakes perform automatic schema validation by checking against a set of rules to determine the compatibility of a write from a DataFrame to a table. One such rule is the existence of all DataFrame columns in the target table. An occurrence of an extra or missing column in the DataFrame raises an exception error. Another rule is that the DataFrame and target table must contain the same column types, which otherwise will raise an exception. Delta Lake also use DDL (Data Definition Language) to add new columns explicitly. This data lake feature helps prevent the ingestion of incorrect data, thereby ensuring high data quality.
    \item[\textbullet] \textbf{Compatibility with Spark's API:} Delta Lake is built on Apache Spark and is fully compatible with Spark API, which helps build efficient and reliable big data pipelines.
    \item[\textbullet] \textbf{Flexibility and integration:} Delta lake is an open-source storage layer and utilizes the Parquet format to store data files, which promotes data sharing and makes it easier to integrate with other technologies and drive innovation.
\end{itemize}

\section{Implementation}
To use Delta Lake interactively within the Spark SQL, Scala, or Python shell, we need a local installation of Apache Spark. Depending on whether we want to use SQL, Python, or Scala, we can set up either the SQL, PySpark, or Spark shell, respectively.

\subsection*{Spark SQL Shell:}

\begin{lstlisting}[language=bash]
    bin/spark-sql --packages io.delta:delta-core_2.12:2.3.0
    --conf "spark.sql.extensions=io.delta.sql.DeltaSparkSessionExtension"
    --conf "spark.sql.catalog.spark_catalog=org.apache.spark.sql.delta.catalog.DeltaCatalog"
\end{lstlisting}

\subsection*{PySpark Shell:}
\begin{enumerate}
    \item Install the PySpark version that is compatible with the Delta Lake version by running the following:
    \begin{lstlisting}[language=bash]
    pip install pyspark==<compatible-spark-version>
    \end{lstlisting}
    \item Run PySpark with the Delta Lake package and additional configurations:
    \begin{lstlisting}[language=bash]
    pyspark --packages io.delta:delta-core_2.12:2.3.0 
    --conf "spark.sql.extensions=io.delta.sql.DeltaSparkSessionExtension" 
    --conf "spark.sql.catalog.spark_catalog=org.apache.spark.sql.delta.catalog.DeltaCatalog"
    \end{lstlisting}
\end{enumerate}

\subsection*{Scala Shell:}
Download the compatible version of Apache Spark by following instructions from Downloading Spark, either using pip or by downloading and extracting the archive and running spark-shell in the extracted directory.

\begin{lstlisting}[language=bash]
    bin/spark-shell --packages io.delta:delta-core_2.12:2.3.0 
    --conf "spark.sql.extensions=io.delta.sql.DeltaSparkSessionExtension" 
    --conf "spark.sql.catalog.spark_catalog=org.apache.spark.sql.delta.catalog.DeltaCatalog"
\end{lstlisting}

\subsection*{Create Table:}
To create a Delta table, write a DataFrame out in the delta format. We can use existing Spark SQL code and change the format from parquet, csv, json, and so on, to delta.

\begin{enumerate}
    \item \textbf{SQL:}
    \begin{lstlisting}[language=sql]
    CREATE TABLE delta.`/tmp/delta-table` USING DELTA AS SELECT col1 as id FROM VALUES 0,1,2,3,4;
    \end{lstlisting}
    \item \textbf{Python:}
    \begin{lstlisting}[language=python]
    data = spark.range(0, 5)
    data.write.format("delta").save("/tmp/delta-table")
    \end{lstlisting}
    \item \textbf{Scala:}
    \begin{lstlisting}[language=scala]
    val data = spark.range(0, 5)
    data.write.format("delta").save("/tmp/delta-table")
    \end{lstlisting}
    \item \textbf{Java:}
    \begin{lstlisting}[language=java]
    import org.apache.spark.sql.SparkSession;
    import org.apache.spark.sql.Dataset;
    import org.apache.spark.sql.Row;
    
    SparkSession spark = ...   // create SparkSession
    
    Dataset<Row> data = spark.range(0, 5);
    data.write().format("delta").save("/tmp/delta-table");
    \end{lstlisting}
\end{enumerate}

These operations create a new Delta table using the schema that was inferred from your DataFrame

\section*{Conclusion}
\addcontentsline{toc}{section}{Conclusion}

Delta Lake is an important tool for big data processing, providing reliable data management and ensuring data integrity at scale. Its ACID transactions, schema enforcement, and data versioning features make it a popular choice for companies that need to process large amounts of data with high accuracy and reliability.

By using Delta Lake, data engineers and data scientists can easily manage data quality, track data lineage, and collaborate on data analysis projects. With its seamless integration with other big data tools. Delta Lake provides a powerful solution for big data processing that can help companies gain insights from their data faster and more efficiently.

    
    \chapter{Technologies used}\label{chap:3}

    \section*{Introduction}
In this chapter, we will examine in detail the key technologies that are used in our solution to provide advanced features and meet the specific needs of our project. The three main technologies we will cover are Delta Lake, Trino, and Microfrontends.

\section{Delta Lake}

Delta Lake is a data management technology that enables efficient and reliable storage, management, and analysis of massive volumes of data. It is built on a file-based Parquet architecture and offers advanced features such as ACID (Atomicity, Consistency, Isolation, Durability) transaction management and compatibility with popular analytics tools. Delta Lake also ensures data integrity, query consistency, and supports replication and recovery in case of failures.

The concept of a `lakehouse' is made possible by Delta Lake. It is a data architecture that combines the benefits of data warehouses and data lakes, providing a unique and consistent approach to data management. Data is stored in Parquet format in the data lake, enabling continuous and batch processing.
\begin{itemize}
\item \textbf{Enables Lakehouse architecture:} Delta Lake enables a continuous and streamlined data architecture that allows organizations to manage and process massive volumes of data in a continuous and batch manner without the hassle of separately managing and operating streaming, data warehouses, and data lakes.
\item \textbf{Enables intelligent data management for data lakes:} Delta Lake provides efficient and scalable metadata management, which provides insights into massive data volumes in data lakes. With this information, data governance and management tasks can be performed more effectively.
\item \textbf{Schema enforcement for improved data quality:} Since data lakes don't have a defined schema, it becomes easy for bad/incompatible data to enter the data systems. Data quality is improved through automatic schema validation, which validates DataFrame and table compatibility before writes.
\item \textbf{Enables ACID transactions:} Most organizational data architectures involve numerous ETL and ELT movements in and out of data storage, which opens it up to more complexity and failure points. Delta Lake ensures data durability and persistence during ETL and other data operations. Delta Lake captures all data changes during data operations in a transaction log, ensuring data integrity and reliability during data operations.
\end{itemize}

\cite{deltalake}

\section{Key Benefits and Features of Delta Lake}
\begin{flushleft}
With Delta Lake, data is stored in an optimized format, such as Parquet, in a data lake. This format enables efficient query processing regardless of the mode of data access, whether it is streaming or batch processing.
\end{flushleft}

\cite{databricksdelta}

\begin{figure}[H]
\centering
\includegraphics[width=\linewidth]{images/delta_lake_architecture.png}
\caption{Delta Lake multi-hop architecture}\label{fig:delta-lake-architecture}
\end{figure}

\begin{itemize}
\item[\textbullet] \textbf{Audit trails and history:} In Delta Lake, each write exists as a transaction and is sequentially recorded in a transaction log. Therefore, all modifications or validations made to the transaction log are recorded, leaving a complete trail to be used for historical audits, versioning, or time travel purposes. This Delta Lake feature ensures data integrity and reliability for enterprise data operations.
\item[\textbullet] \textbf{Time travel and data versioning:} Since each write creates a new version and stores the old version in the transaction log, users can view/restore old versions of data by providing the timestamp or version number of an existing table or directory to the Spark read API. Using the provided version number, Delta Lake then constructs a complete snapshot of the version with the information provided by the transaction log. Rollbacks and version management play a crucial role in machine learning experimentation, where data scientists iteratively modify hyperparameters to train models and can revert to previous changes if necessary.
\item[\textbullet] \textbf{Unifies batch and stream processing:} Each table in a Delta Lake is both a batch and stream sink. With Structured Streaming in Spark, organizations can efficiently stream and process data. Additionally, with efficient metadata management, scalability, and ACID quality for each transaction, near-real-time analytics becomes possible without using a more complicated two-tier data architecture.
\item[\textbullet] \textbf{Efficient and scalable metadata management:} Delta Lake stores metadata information in the transaction log and leverages the distributed processing power of Spark to quickly process, read, and manage large volumes of data metadata, thereby enhancing data governance.
\item[\textbullet] \textbf{ACID transactions:} Delta Lake ensures that users always see a consistent view of data in a table or directory. It achieves this by capturing every modification made in a transaction log and isolating it at the strongest isolation level, the serializable level. At the serializable level, every existing operation has and follows a serial sequence that, when executed one by one, provides the same result as stated in the table.
\item[\textbullet] \textbf{Data Manipulation Language (DML) operations:} Delta Lake supports DML operations such as updates, deletes, and merges, which play a role in complex data operations such as Change Data Capture (CDC), continuous upserts, and Slowly Changing Dimensions (SCD). Operations like CDC ensure data synchronization across all data systems and minimize time and resources spent on ELT operations. For example, using CDC, instead of ETL-ing all available data, only the recently updated data since the last operation undergoes transformation.
\item[\textbullet] \textbf{Schema enforcement:} Delta Lake performs automatic schema validation by checking a set of rules to determine the compatibility of a DataFrame write to a table. One such rule is the existence of all DataFrame columns in the target table. An occurrence of an extra or missing column in the DataFrame raises an error exception. Another rule is that the DataFrame and the target table must have the same column types, which, if not, triggers an exception. Delta Lake also uses Data Definition Language (DDL) to explicitly add new columns. This data lake feature helps avoid the ingestion of incorrect data, ensuring high data quality.
\item[\textbullet] \textbf{Compatibility with Spark API:} Delta Lake is built on Apache Spark and is fully compatible with the Spark API, enabling the creation of efficient and reliable large-scale data pipelines.
\item[\textbullet] \textbf{Flexibility and integration:} Delta Lake is an open-source storage layer and utilizes the Parquet format for storing data files, which promotes data sharing and facilitates integration with other technologies, fostering innovation.
\end{itemize}

\section{Trino}

Trino, formerly known as Presto, is a distributed, open-source SQL query engine. It is designed to execute interactive and analytical queries at a large scale on heterogeneous and distributed data. Trino offers great versatility by allowing access to various types of data sources, whether they are relational databases, file systems, real-time data sources, or cloud storage services. With its distributed design, Trino enables high performance and horizontal scalability, making it an essential tool for data analysis in our solution.

\cite{trino}

\begin{figure}[H]
\centering
\includegraphics[width=0.8\linewidth]{images/trino_architecture.png}
\caption{Overview of Trino architecture with coordinator and workers}\label{fig:trino-architecture}
\end{figure}

\begin{enumerate}
\item A coordinator is a Trino server that handles incoming queries and manages workers to execute the queries.
\item A worker is a Trino server responsible for executing tasks and processing data.
\item The discovery service typically runs on the coordinator and allows workers to register to participate in the cluster.
\item All communications and data transfers between clients, the coordinator, and workers use REST-based interactions over HTTP/HTTPS.
\end{enumerate}

\section{Spring Boot}

Spring Boot is an open-source framework for Java application development. It provides a simplified and opinionated approach to creating standalone, production-ready Java applications without the need for complex configuration.

One of the main advantages of Spring Boot is its ability to reduce boilerplate configuration and simplify application development by providing smart default configuration definitions and automating many development tasks. It also embeds an application server, making it easy to deploy and run the application without requiring an external application server.

Spring Boot follows the annotation-driven programming paradigm, where annotations are used to configure and orchestrate different parts of the application. It offers a wide range of features, such as dependency injection, externalized configuration, error handling, security, data access, etc. These features are bundled into starters, which are pre-defined dependencies that facilitate adding specific functionality to the application.

With its simplified approach, Spring Boot allows developers to focus more on the business logic of their application rather than tedious configuration tasks. It also promotes good development practices, such as separation of concerns and modularity, making applications more maintainable and scalable.

\cite{springboot}

\section{Keycloak}

Keycloak is an open-source Identity and Access Management (IAM) solution developed by Red Hat. It provides comprehensive features for user management, authentication, authorization, and securing applications.

Keycloak helps centralize and simplify identity management within an IT infrastructure. It offers features such as user registration, multi-factor authentication, role and permission management, session management, and integration with common authentication and authorization protocols like OAuth 2.0 and OpenID Connect.

Keycloak provides functionality for managing roles, administrators, users, and passwords. Here's how Keycloak addresses these aspects:

\cite{keycloak}

\begin{enumerate}
\item Keycloak allows defining roles at the realm or application level. Roles can be created and assigned to users to define their permissions and access.
\item Keycloak administrators can create, manage, and assign roles to users through the administration interface or management API.
\item Roles can be used to control access to features, pages, and resources within the application.
\item Keycloak provides specific administration roles like "admin" or "superadmin" that allow users to perform administrative tasks such as managing clients, users, roles, etc.
\item Users can log in using their credentials (username and password) or other supported authentication methods like two-factor authentication, OAuth 2.0, etc.
\item Keycloak supports user-based authentication and provides a registration interface to allow users to create their accounts.
\item Keycloak also offers advanced authentication features such as two-factor authentication, social authentication (via identity providers like Google, Facebook, etc.), and certificate-based authentication.
\end{enumerate}

\begin{figure}[H]
\centering
\includegraphics[width=\linewidth]{images/Keycloak-overview-screenshot.png}
\caption{Keycloak Overview}\label{fig:keycloak}
\end{figure}


\section{Kafka}

Kafka is a distributed and scalable data streaming platform designed to efficiently handle the transmission and processing of real-time data streams. It was developed by Apache Software Foundation.

Kafka is based on a distributed log architecture, where data is stored as streams of messages in "topics". Data producers send messages to specific "topics," while consumers subscribe to those "topics" to retrieve the messages. This allows for asynchronous communication and clear separation between data producers and consumers.

\cite{kafka}

The key features of Kafka include:

\begin{enumerate}
\item Scalability: Kafka is designed to handle large volumes of data and can be horizontally scaled to meet growing performance needs. It can handle high workloads and process thousands of messages per second.
\item Fault tolerance: Kafka ensures high availability and fault tolerance by replicating data across multiple nodes in the cluster. This ensures data reliability and availability even in case of node failures.
\item Data durability: Messages stored in Kafka are persistent and can be retained for a defined period. This allows for message replay and data recovery when needed, which is crucial for use cases requiring long-term data retention.
\item Stream processing: Kafka is designed for real-time stream processing. It enables applications to consume continuous data streams and process them in real-time, which is critical for use cases requiring real-time analysis, data pipelines, etc.
\item Integration with other tools: Kafka easily integrates with other tools and frameworks such as Spark, Hadoop, Flink, etc. This enables seamless integration with the Big Data ecosystem and facilitates data ingestion, processing, and streaming.
\end{enumerate}

\begin{figure}[H]
\centering
\includegraphics[width=\linewidth]{images/kafka.jpg}
\caption{Kafka Architecture}\label{fig:kafka}
\end{figure}

\section*{Conclusion}
these technologies, developers and data professionals can leverage their strengths to build powerful and scalable applications. Trino enables fast and flexible data querying, Spring Boot simplifies application development, Keycloak provides robust IAM capabilities, and Kafka enables real-time data streaming and processing.
\fi
 
% \include{perspectives}
%%conclusion
\conclusion
Le stage a été une expérience enrichissante qui a permis d'explorer divers aspects des microservices en utilisant des technologies telles que Delta Lake, Trino, Spring Boot, et Keycloak. L'environnement de travail était propice à l'apprentissage et à la mise en pratique de ces concepts.

L'adoption des microservices présente de nombreux avantages par rapport à une architecture monolithique. Les microservices offrent une meilleure scalabilité et flexibilité, permettant le déploiement, le développement et la mise à l'échelle indépendants de chaque service. De plus, la communication entre les microservices via des API facilite l'intégration et la collaboration entre les différentes parties du système.

Pendant le stage, nous avons appris à concevoir et implémenter des microservices en utilisant Spring Boot, en exploitant ses fonctionnalités de persistence, de sécurité, et de création d'API REST. Nous avons également intégré Keycloak pour gérer l'authentification et l'autorisation des utilisateurs dans notre architecture de microservices. L'utilisation de Delta Lake a permis de garantir la fiabilité des données et de faciliter la gestion des mises à jour.
% 
\lhead[]{} \rhead[]{} \chead[]{}

%%biblio
% \addcontentsline{toc}{chapter}{Bibliographie}
% \bibliographystyle{abbrv}
% \bibliography{biblio}


%\include{annexe}



\end{document}          
