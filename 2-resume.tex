\resume
%\selectlanguage{french}

Le présent rapport de projet de fin d'études (PFE) met en évidence les différentes étapes de notre projet de transformation de l'architecture monolithique en Groovy vers une architecture basée sur des microservices en Java, tout en effectuant une transition de AngularJS vers React et en adoptant une approche de microfrontend. Nous avons entrepris cette démarche afin d'améliorer la flexibilité, la maintenabilité, les performances, la cohérence des données et la modularité de notre application.
\medskip

Initialement, notre code monolithique était développé en Groovy. La première étape de notre projet a consisté à analyser cette architecture monolithique et à identifier les composants qui pourraient être isolés en microservices indépendants. Parallèlement, nous avons évalué les avantages de migrer de AngularJS, un framework JavaScript obsolète, vers React, un framework plus moderne et performant.

\medskip
Nous avons procédé à la conception et à la mise en œuvre de ces microservices en utilisant SpringBoot, en veillant à découpler les fonctionnalités et à assurer une communication efficace entre les services. Dans le même temps, nous avons migré progressivement notre code AngularJS vers React, en réécrivant les fonctionnalités existantes avec les meilleures pratiques de développement React.
\medskip

Pour assurer la cohérence des données entre les microservices, nous avons mis en place Delta Lake, une technologie de gestion de données incrémentielle. Delta Lake a permis de garantir la fiabilité des données et de simplifier les opérations de lecture et d'écriture sécurisées entre les services.

\medskip
Les connecteurs en Java ont facilité la communication entre le frontend basé sur React et Delta Lake, assurant ainsi une connexion robuste et sécurisée. Les connecteurs ont également permis des opérations de lecture et d'écriture efficaces, en garantissant l'intégrité et la cohérence des données.

\medskip
 
En parallèle, nous avons également adopté une approche de microfrontend pour notre architecture frontend. Cela nous a permis de découpler les fonctionnalités du frontend en modules autonomes, offrant ainsi une plus grande flexibilité et la possibilité de les développer et de les déployer indépendamment.
\medskip

Enfin, pour faciliter le déploiement et la gestion de notre architecture basée sur des microservices, React et le microfrontend, nous avons adopté l'utilisation de Kubernetes et Docker. Ces outils ont permis l'orchestration et la mise à l'échelle efficaces des services, tout en simplifiant le déploiement dans différents environnements.

\noindent\rule[2pt]{\textwidth}{0.5pt}
{\textbf{Mots-clés:}}
Monolithique, Groovy, architecture microservices, AngularJS, React, Delta Lake, Microfrontend, Kubernetes, Docker.
\\
\noindent\rule[2pt]{\textwidth}{0.5pt}

\clearpage

