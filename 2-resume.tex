\resume
%\selectlanguage{french}






Ce rapport fournit une analyse approfondie du projet complexe entrepris pour faire évoluer la source de données de Izicap. Le projet a nécessité la suppression de MariaDB et la mise en place de Delta Lake, une solution de stockage et de traitement de données hautes performances. De plus, l'architecture monolithique existante a été divisée en microservices utilisant Spring Boot comme backend avec un connecteur approprié pour Delta Lake, et les micro-frontends ReactJS comme frontend.

\medskip

Le projet présentait de nombreux défis qui nécessitaient l'application de technologies et de stratégies avancées. Celles-ci comprenaient la migration des données, la garantie de la cohérence et de l'exactitude des données, la gestion de la complexité des systèmes distribués et l'intégration de diverses technologies et services. Le rapport décrit les différentes solutions développées pour relever ces défis, notamment l'utilisation d'algorithmes de traitement de données avancés, d'architectures informatiques distribuées et de la conteneurisation.

\medskip

Malgré les défis rencontrés, le projet reste un travail en cours. Le rapport donne un aperçu des efforts en cours pour améliorer l'évolutivité, les performances et la fonctionnalité globale du projet.

\vspace{1cm}


\noindent\rule[2pt]{\textwidth}{0.5pt}

{\textbf{Mots clés :}}
Projet, Source de données, Delta Lake, Monolith, Microservice, Spring Boot, ReactJS, Micro-Frontend, Back-End, Front-End, Données, Évolutivité, Performance.
\\
\noindent\rule[2pt]{\textwidth}{0.5pt}

\clearpage

