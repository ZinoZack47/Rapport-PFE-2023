\introduction
Dans un monde numérique en constante évolution, les entreprises sont confrontées à des défis majeurs pour maintenir leurs applications à la pointe de la technologie et répondre aux attentes croissantes des utilisateurs. L'architecture monolithique présente des limitations en termes de flexibilité, de maintenabilité, de performances et de modularité, ce qui pousse de nombreuses organisations à adopter des architectures basées sur des microservices.

\medskip

Ce rapport met en évidence la démarche entreprise pour transformer une architecture monolithique en Groovy vers une architecture basée sur des microservices en utilisant Java, tout en migrant de AngularJS vers React et en adoptant une approche de microfrontend. L'objectif principal de cette transformation était d'améliorer la flexibilité, la maintenabilité, les performances, la cohérence des données et la modularité de l'application.

\medskip

La première étape a été d'analyser l'architecture monolithique existante et d'identifier les composants pouvant être isolés en microservices indépendants. En parallèle, une évaluation des avantages de migrer de AngularJS vers React a été réalisée.

\medskip

La conception et la mise en œuvre des microservices ont été réalisées en utilisant Spring-Boot, en veillant à découpler les fonctionnalités et à assurer une communication efficace entre les services. Le code AngularJS a été progressivement migré vers React en utilisant les meilleures pratiques de développement React.

\medskip

Pour garantir la cohérence des données entre les microservices, la technologie de gestion de données incrémentielle Delta Lake a été utilisée. Des connecteurs en Java ont facilité la communication entre le frontend basé sur React et Delta Lake, assurant une connexion robuste et sécurisée.

\medskip

En parallèle, une approche de microfrontend a été adoptée pour le frontend, permettant de découpler les fonctionnalités en modules autonomes.

\medskip

Ce rapport détaillera chaque étape de la transformation, mettant l'accent sur les décisions prises, les défis rencontrés et les résultats obtenus. Il soulignera les avantages de l'architecture basée sur des microservices, l'utilisation de React et du microfrontend, ainsi que l'utilisation de Kubernetes et Docker. Cette étude de cas offre une perspective précieuse sur la modernisation des applications et les meilleures pratiques de développement dans un environnement en constante évolution.