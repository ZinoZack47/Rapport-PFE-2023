\addcontentsline{toc}{section}{Introduction}

\section*{Introduction}

L'informatique décisionnelle, également connue sous le terme de Business Intelligence (BI), englobe les processus, les technologies et les outils utilisés pour collecter, stocker, analyser et présenter les données dans le but de soutenir la prise de décision et d'aider les entreprises à obtenir des informations exploitables. L'informatique décisionnelle implique la transformation des données brutes en informations significatives et en connaissances exploitables pour les décideurs. On va voir dans ce chaptire pourquoi on a decidé d'opter pour Delta Lake au lieu de ses contre-parts.

\section{Différence entre un data warehouse, un data lake, un datalakehouse et un delta lake}
\begin{enumerate}
    \item[$\bullet$] \textbf{Data Warehouse:} Un data warehouse est une base de données centralisée qui est spécifiquement conçue pour le reporting et l'analyse. Il stocke les données structurées provenant de différentes sources, les organise selon un modèle de données prédéfini et les optimise pour des requêtes analytiques. Les données dans un data warehouse sont généralement cohérentes, intégrées et historisées. Cependant, la construction et la maintenance d'un data warehouse peuvent être complexes et coûteuses.
    \item[$\bullet$] \textbf{Data Lake:} Un data lake est un référentiel de données centralisé qui stocke de grandes quantités de données brutes, structurées et non structurées. Contrairement au data warehouse, le data lake ne nécessite pas une modélisation préalable des données. Il offre une grande flexibilité et évolutivité pour stocker des données hétérogènes. Cependant, l'intégration et la qualité des données peuvent être des défis dans un data lake.
    \item[$\bullet$] \textbf{Datalakehouse:} Le datalakehouse est une architecture émergente qui combine les avantages du data warehouse et du data lake. Il permet de stocker et de traiter à la fois des données brutes et des données structurées dans un environnement centralisé. Cette approche hybride offre la flexibilité d'un data lake et la capacité d'analyse d'un data warehouse. Cependant, la mise en place d'un datalakehouse peut nécessiter des efforts supplémentaires pour garantir la qualité des données et l'efficacité des requêtes.
    \item[$\bullet$] \textbf{Delta Lake:} Delta Lake est une technologie qui s'intègre aux data lakes existants pour fournir des fonctionnalités supplémentaires, telles que la gestion des transactions ACID (Atomicité, Cohérence, Isolation, Durabilité), la gestion des mises à jour incrémentielles et la garantie de la cohérence des données. Delta Lake est construit sur Apache Parquet et Apache Arrow, ce qui permet d'accélérer les requêtes analytiques et d'améliorer les performances globales. Cependant, l'utilisation de Delta Lake peut nécessiter des compétences techniques supplémentaires et peut avoir un impact sur la complexité de l'architecture de données. 
\end{enumerate}

\section{Avantages et inconvénients de chaque architecture}

\subsection{Data Warehouse}
\textbf{Avanatges:}
\begin{enumerate}
    \item Données cohérentes et intégrées
    \item Modélisation préalable des données pour une analyse optimisée
    \item Hautes performances pour les requêtes analytiques
\end{enumerate}

\textbf{Inconvénients:}
\begin{enumerate}
    \item Coût élevé de construction et de maintenance
    \item Complexité de la modélisation des données
    \item Limitations pour l'intégration de données non structurées
\end{enumerate}


\subsection{Data Lake}
\textbf{Avanatges:}
\begin{enumerate}
    \item Stockage économique de grandes quantités de données
    \item Flexibilité pour intégrer des données brutes et non structurées
    \item Capacité à traiter des données de différentes sources
\end{enumerate}

\textbf{Inconvénients:}
\begin{enumerate}
    \item Difficulté à maintenir la qualité des données et la gouvernance
    \item Besoin d'outils avancés pour l'analyse et le traitement des données
    \item Requiert des compétences techniques pour l'exploitation efficace des données
\end{enumerate}

\subsection{Datalakehouse}
\textbf{Avanatges:}
\begin{enumerate}
    \item Combinaison des avantages du data warehouse et du data lake
    \item Flexibilité pour stocker et analyser des données brutes et structurées
    \item Possibilité d'évoluer en fonction des besoins évolutifs
\end{enumerate}

\textbf{Inconvénients:}
\begin{enumerate}
    \item Nécessite des efforts supplémentaires pour la qualité des données
    \item Complexité accrue de l'architecture de données
    \item Besoin de compétences techniques pour la mise en place et la gestion
\end{enumerate}

\subsection{Delta Lake}
\textbf{Avanatges:}
\begin{enumerate}
    \item Gestion des transactions ACID pour une cohérence des données
    \item Prise en charge des mises à jour incrémentielles et du traitement des flux de données
    \item Hautes performances pour les requêtes analytiques
\end{enumerate}

\textbf{Inconvénients:}
\begin{enumerate}
    \item Nécessite des compétences techniques spécifiques
    \item Impact sur la complexité de l'architecture de données existante
    \item Peut nécessiter des adaptations pour une intégration transparente avec les outils existants
\end{enumerate}

\section*{Conclusion}
\addcontentsline{toc}{section}{Conclusion}
Lors de l'évaluation des différentes architectures de données pour Izicap, il est important de comprendre les besoins spécifiques liés à la gestion des fichiers bancaires, des reçus de transactions et des opérations d'agrégation.

Un data warehouse aurait pu être une option envisageable, offrant des structures de données organisées et optimisées pour les requêtes analytiques. Cependant, le principal inconvénient d'un data warehouse réside dans sa nature statique, qui nécessite une modélisation préalable des données et une transformation rigide avant leur chargement. Cela peut poser des défis lors de l'intégration de nouveaux types de fichiers ou de l'évolution des besoins en matière d'agrégation.

D'autre part, un data lake présente des avantages en termes de stockage économique et de flexibilité pour intégrer des données brutes et non structurées. Cependant, il peut être plus complexe de maintenir la qualité des données et la gouvernance, et des compétences techniques spécifiques sont nécessaires pour exploiter efficacement les données du data lake.

En conclusion, Delta Lake a été privilégié en raison de sa capacité à répondre aux besoins spécifiques d'Izicap en matière de gestion des fichiers bancaires, des reçus de transactions et des opérations d'agrégation. Il offre la flexibilité et la performance nécessaires tout en maintenant l'intégrité des données, ce qui en fait un choix solide pour l'architecture de données de l'entreprise.