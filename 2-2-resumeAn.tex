\resumeAn
%\begin{abstract}
This final year project report highlights the different stages of our project to transform a Groovy-based monolithic architecture into a Java-based microservices architecture, while transitioning from AngularJS to React and adopting a microfrontend approach. Our goal was to enhance the flexibility, maintainability, performance, data consistency, and modularity of our application.
\medskip

Initially, our codebase was developed as a monolith using Groovy. The first step of our project involved analyzing the monolithic architecture and identifying components that could be isolated into independent microservices. Concurrently, we evaluated the benefits of migrating from the outdated AngularJS framework to the more modern and performant React framework.

\medskip
We proceeded with the design and implementation of these microservices using Spring Boot, ensuring loose coupling between functionalities and effective communication among services. Simultaneously, we gradually migrated our AngularJS code to React, rewriting existing features according to React's best development practices.  
\medskip

To ensure data consistency among microservices, we implemented Delta Lake, an incremental data management technology. Delta Lake guaranteed reliable data and simplified secure read and write operations between services. Java connectors facilitated communication between the React-based frontend and Delta Lake, ensuring robust and secure connectivity. These connectors also enabled efficient read and write operations, ensuring data integrity and consistency.

\medskip
In parallel, we adopted a microfrontend approach for our frontend architecture, decoupling frontend features into autonomous modules. This provided greater flexibility and the ability to develop and deploy modules independently.

\medskip

Finally, to facilitate deployment and management of our microservices, React, and microfrontend architecture, we adopted Kubernetes and Docker. These tools enabled efficient orchestration, scaling of services, and simplified deployment across different environments.
\medskip

In conclusion, our project resulted in significant improvements in flexibility, maintainability, performance, data consistency, and modularity of our application. The transition from a Groovy-based monolithic architecture to a Java-based microservices architecture, combined with the migration from AngularJS to React, the adoption of Delta Lake, Java connectors, microfrontend, Kubernetes, and Docker, collectively optimized our system.

    \noindent\rule[2pt]{\textwidth}{0.5pt}
    {\textbf{Mots-clés :}}
    Monolithique, Groovy, Microservices, AngularJS, React, Delta Lake, Microfrontend, Kubernetes, Docker\\
    \noindent\rule[2pt]{\textwidth}{0.5pt}
\clearpage