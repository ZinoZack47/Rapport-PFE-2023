%\resumeAr
\setcode{utf8}

\chapter*{\hfill \RL{ملخص}}




\setstretch{1.3}
\begin{flushright}
\RL{
يقدم هذا التقرير تحليلاً متعمقًا للمشروع المعقد الذي تم إجراؤه لتوسيع نطاق مصدر بيانات الشركة. يتطلب المشروع إزالة \LR{MariaDB} وتنفيذ \LR{Delta Lake} ، وهو حل تخزين ومعالجة بيانات عالي الأداء. بالإضافة إلى ذلك ، تم تقسيم المونوليث الحالي إلى خدمات مصغرة باستخدام \LR{Spring Boot} كواجهة خلفية مع موصل مناسب لـ \LR{Delta Lake} ، وواجهة \LR{ReactJS} كواجهة أمامية.
قدم المشروع العديد من التحديات التي تطلبت تطبيق التقنيات والاستراتيجيات المتقدمة. وشمل ذلك ترحيل البيانات ، وضمان اتساق البيانات ودقتها ، وإدارة تعقيدات الأنظمة الموزعة ، ودمج التقنيات والخدمات المختلفة. يحدد التقرير الحلول المختلفة التي تم تطويرها لمواجهة هذه التحديات ، بما في ذلك استخدام خوارزميات معالجة البيانات المتقدمة ، وبنى الحوسبة الموزعة ، والحاويات.
على الرغم من التحديات التي واجهنها ، لا يزال المشروع قيد التنفيذ. يقدم التقرير رؤى حول الجهود الجارية لتحسين قابلية المشروع للتوسع والأداء والوظائف العامة
}
\end{flushright}


\vspace{1cm}


\noindent\rule[2pt]{\textwidth}{0.5pt}
\begin{flushright}
\RL{\textbf{
كلمات مفتاحية \LR{:} } المشروع، مصدر البيانات، \LR{Delta Lake}، المونوليث، الميكروسيرفس، \LR{Springboot}، \LR{ReactJS}، الميكرو-واجهات، الخلفية، الواجهة الأمامية، البيانات، القابلية للتوسع، الأداء. \LR{Delta Lake}، \LR{Springboot}، \LR{ReactJS}، الميكرو-واجهـات. 
}

\end{flushright}
\noindent\rule[2pt]{\textwidth}{0.5pt}

%\end{abstract}






