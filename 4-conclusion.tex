\conclusion
Le stage a été une expérience enrichissante qui a permis d'explorer divers aspects des microservices en utilisant des technologies telles que Delta Lake, Trino, Spring Boot, et Keycloak. L'environnement de travail était propice à l'apprentissage et à la mise en pratique de ces concepts.

L'adoption des microservices présente de nombreux avantages par rapport à une architecture monolithique. Les microservices offrent une meilleure scalabilité et flexibilité, permettant le déploiement, le développement et la mise à l'échelle indépendants de chaque service. De plus, la communication entre les microservices via des API facilite l'intégration et la collaboration entre les différentes parties du système.

Pendant le stage, nous avons appris à concevoir et implémenter des microservices en utilisant Spring Boot, en exploitant ses fonctionnalités de persistence, de sécurité, et de création d'API REST. Nous avons également intégré Keycloak pour gérer l'authentification et l'autorisation des utilisateurs dans notre architecture de microservices. L'utilisation de Delta Lake a permis de garantir la fiabilité des données et de faciliter la gestion des mises à jour.