\introduction

In a constantly evolving digital world, businesses face major challenges in keeping their applications at the forefront of technology and meeting the growing expectations of users. Monolithic architecture has limitations in terms of flexibility, maintainability, performance, and modularity, which has led many organizations to adopt microservices-based architectures.

\medskip

This report highlights the approach taken to transform a monolithic Groovy architecture into a microservices-based architecture using Java, while migrating from AngularJS to React and adopting a microfrontend approach. The main goal of this transformation was to improve the flexibility, maintainability, performance, data consistency, and modularity of the application.

\medskip

The first step was to analyze the existing monolithic architecture and identify components that could be isolated into independent microservices. In parallel, an assessment of the benefits of migrating from AngularJS to React was conducted.

\medskip

The design and implementation of the microservices were done using Spring Boot, ensuring decoupling of functionalities and ensuring efficient communication between services. The AngularJS code was gradually migrated to React using best practices in React development.

\medskip

To ensure data consistency among the microservices, the incremental data management technology Delta Lake was used. Java connectors facilitated communication between the React-based frontend and Delta Lake, ensuring a robust and secure connection.

\medskip

In parallel, a microfrontend approach was adopted for the frontend, allowing functionalities to be decoupled into autonomous modules.

\medskip

This report will detail each step of the transformation, focusing on the decisions made, challenges encountered, and results achieved. It will highlight the benefits of microservices-based architecture, the use of React and microfrontend, as well as the use of Kubernetes and Docker. This case study provides valuable insights into application modernization and best development practices in a constantly changing environment.